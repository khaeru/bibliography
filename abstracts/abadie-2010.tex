 Building on an idea in Abadie and Gardeazabal (2003), this article investigates the application of synthetic control methods to comparative case studies. We discuss the advantages of these methods and apply them to study the effects of Proposition 99, a large-scale tobacco control program that California implemented in 1988. We demonstrate that, following Proposition 99, tobacco consumption fell markedly in California relative to a comparable synthetic control region. We estimate that by the year 2000 annual per-capita cigarette sales in California were about 26 packs lower than what they would have been in the absence of Proposition 99. Using new inferential methods proposed in this article, we demonstrate the significance of our estimates. Given that many policy interventions and events of interest in social sciences take place at an aggregate level (countries, regions, cities, etc.) and affect a small number of aggregate units, the potential applicability of synthetic control methods to comparative case studies is very large, especially in situations where traditional regression methods are not appropriate. 