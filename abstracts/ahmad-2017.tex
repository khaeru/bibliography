Energy-efficient urbanization and public health pose major development challenges for India. While both issues are intensively studied, their interaction is not well understood. Here we explore the relationship between urban infrastructures, public health, and household-related emissions, identifying potential synergies and trade-offs of specific interventions by analyzing nationally representative household surveys from 2005 and 2012. Our analysis confirms previous characterizations of the environmental-health transition, but also points to an important role of energy use and urbanization as modifiers of this transition. We find that non-motorized transport may prove a sweet spot for development, as its use is associated with lower emissions and better public health in cities. Urbanization and improved access to basic services correlate with lower short-term morbidity (STM), such as fever, cough and diarrhea. Our analysis suggests that a 10% increase in urbanization from current levels and concurrent improvement in access to modern cooking and clean water could lower STM for 2.4 million people. This would be associated with a modest increase in electricity related emissions of 84 ktCO 2 e annually. Promoting energy-efficient mobility systems, for instance by a 10% increase in bicycling, could lower chronic conditions like diabetes and cardio-vascular diseases for 0.3 million people while also abating emissions. These findings provide empirical evidence to validate that energy-efficient and sustainable urbanization can address both public health and climate change challenges simultaneously.