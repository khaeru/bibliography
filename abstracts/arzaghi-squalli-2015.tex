In recent years, governments that have historically subsidized domestic fuel consumption face an ever-growing challenge in maintaining fuel subsidies and have embarked on subsidy reform. This paper estimates the price and income elasticity of demand for gasoline in countries where fuel prices are government-subsidized. We make use of biennial panel road-sector data for 32 countries over the 1998–2010 period and find demand for gasoline to be price inelastic both in the short run and long run. We estimate the short-run price and income elasticities at −0.05 and 0.16 and the long-run price and income elasticities at −0.25 and 0.81, respectively. It is our contention that concerned governments should play an active role in identifying and committing to a road map to progressively abandoning fuel subsidies. They should also not be discouraged by relatively small consumption corrections in the short run. A reduction in subsidies can eventually release considerable amount of resources for more crucial and potentially growth-enhancing public services such as education and health.