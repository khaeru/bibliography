This paper develops a multi-regional general equilibrium model for climate policy analysis based on the latest version of the MIT Emissions Prediction and Policy Analysis (EPPA) model. We develop two versions so that we can solve the model either as a fully inter-temporal optimization problem (forward-looking, perfect foresight) or recursively. The standard EPPA model on which these models are based is solved recursively, and it is necessary to simplify some aspects of it to make inter-temporal solution possible. The forward-looking capability allows one to better address economic and policy issues such as borrowing and banking of GHG allowances, efficiency implications of environmental tax recycling, endogenous depletion of fossil resources, international capital flows, and optimal emissions abatement paths among others. To evaluate the solution approaches, we benchmark each version to the same macroeconomic path, and then compare the behavior of the two versions under a climate policy that restricts greenhouse gas emissions. We find that the energy sector and CO$_2$ price behavior are similar in both versions (in the recursive version of the model we force the inter-temporal theoretical efficiency result that abatement through time should be allocated such that the CO$_2$ price rises at the interest rate.) The main difference that arises is that the macroeconomic costs are substantially lower in the forward-looking version of the model, since it allows consumption shifting as an additional avenue of adjustment to the policy. On the other hand, the simplifications required for solving the model as an optimization problem, such as dropping the full vintaging of the capital stock and fewer explicit technological options, likely have effects on the results. Moreover, inter-temporal optimization with perfect foresight poorly represents the real economy where agents face high levels of uncertainty that likely lead to higher costs than if they knew the future with certainty. We conclude that while the forward-looking model has value for some problems, the recursive model produces similar behavior in the energy sector and provides greater flexibility in the details of the system that can be represented.
