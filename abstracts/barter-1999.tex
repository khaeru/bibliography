Recent decades have seen rapid change in the urban transport of many eastern Asian cities. Some cities have been hailed internationally as transport success stories while others have become known for their intense traffic crises to the extent of threatening both their environmental qualities and economic performance. Accordingly, there is great interest in better understanding urban transport patterns in the large cities of eastern Asia. However, the literature lacks clear internationally comparable information on these cities and their transport systems. A review of the literature on land use and transport in Asian cities reveals many misunderstandings and inaccurate interpretations of the current situation in these cities. This thesis attempts to redress this lack of sound urban data and to improve policy interpretations by focusing on nine major cities in Pacific Asia (Bangkok, Hong Kong, Jakarta, Kuala Lumpur, Manila, Seoul, Singapore, Surabaya and Tokyo). The study provides an international comparative perspective on these cities using a large set of data on urban transport, land use and economic factors, as part of a wider study on 46 international cities. A historical review of transport and urban development between 1900 and the 1960s found that, by the end of the period, most of the Asian cities were more vulnerable to problems from an influx of private vehicles than Western cities had been at the equivalent stage in their motorisation. This greater vulnerability was primarily due to higher densities and greater dependence on road-based public transport in most Asian cities, which could be described as ?bus cities?, an archetype that is developed in the thesis. This archetype is found to be useful in better understanding Asian cities in relation to more Western-based theories of city evolution based on the dominant transport technology, as well as helping to interpret past and present transport problems. Analysis of comparative transport and other data for 1990 found that the Asian cities in the sample generally had much lower levels of private vehicle use than European, Canadian, Australian and American cities in the international sample. This is in line perhaps with general expectations, though not without significant variations within the group. The Asian cities also generally had greater roles for public transport and non-motorised transport and much higher urban densities than cities in the other regions, though variations were again significant. A detailed investigation of the special opportunities and challenges for transport of the highdensity urban forms of most of the Asian cities reveals new insights on the root causes of transport problems in such cities. High density offers the opportunity to foster successful public transport and non-motorised accessibility. However, it also means that very high levels of motorised traffic per unit of land area (and hence intense traffic impacts) can emerge quickly, even if vehicle use per capita remains low. Traffic congestion can also emerge rapidly as dense cities motorise. This is a result, not just of poorly developed road systems, but of the fact that road capacity per capita is inherently low in dense cities. This research thus challenges notions in the literature that congestion problems in Asian cities can be solved by road expansion. It establishes, through sound comparative urban data, that there are inherent limits to road provision in dense cities. Contrasting urban transport strategies or models were identified within the Asian sample of cities. In particular, upper-middle-income cities, Bangkok and Kuala Lumpur, were shown to have experienced very rapid motorisation and to have had little success in increasing the relative roles of public transport and non-motorised modes. These trends have led to a severe mismatch between emerging car and motorcycle-oriented transport patterns and the pre-existing highdensity urban form, especially in Bangkok. This ?unrestrained motorisation? model is contrasted with the experiences of wealthier Seoul, Singapore, Hong Kong and Tokyo, which have all restrained and slowed the pace of motorisation to some extent and enhanced the role of public transport. In all four cities, 1990 levels of motorisation and vehicle use were low relative to their levels of income. This ?restraint? model takes advantage of the transport opportunities that are inherent in existing dense urban forms while avoiding many of the problems. It is also shown to have encouraged, or complemented, the evolution of public transport-oriented patterns of urban development. Jakarta, Surabaya and Manila face the choice of following either of these models, but appear more likely to follow Bangkok and Kuala Lumpur, unless policy changes are made. The study then reviews key choices and policies in urban transport in the nine Asian cities over recent decades. It identifies which have been most decisive in defining the models ?chosen? by each city. Although many decisions are important, the thesis argues that a particularly crucial choice is the decision of whether or not to restrain private vehicle ownership and use. The Asian cities following the ?restraint? model began to restrain private vehicles at an early stage in their motorisation and generally well before they had developed high-quality or high-capacity public transport systems. This challenges the common view that a city must already have a first-class public transport system before traffic restraint can be effective or politically acceptable. In fact, this study suggests that early introduction of traffic restraint can facilitate the gradual development of well-functioning transport systems, including mass transit systems. Insights drawn from the results of this study potentially have important implications for transport and urban policy debates in low-income and middle-income cities everywhere, particularly those that are beginning to motorise quickly from previously low levels of vehicle ownership.
