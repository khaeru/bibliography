The Paris Agreement does not only stipulate to limit the global average temperature increase to well below 2 °C, it also calls for ‘making finance flows consistent with a pathway towards low greenhouse gas emissions’. Consequently, there is an urgent need to understand the implications of climate targets for energy systems and quantify the associated investment requirements in the coming decade. A meaningful analysis must however consider the near-term mitigation requirements to avoid the overshoot of a temperature goal. It must also include the recently observed fast technological progress in key mitigation options. Here, we use a new and unique scenario ensemble that limit peak warming by construction and that stems from seven up-to-date integrated assessment models. This allows us to study the near-term implications of different limits to peak temperature increase under a consistent and up-to-date set of assumptions. We find that ambitious immediate action allows for limiting median warming outcomes to well below 2 °C in all models. By contrast, current nationally determined contributions for 2030 would add around 0.2 °C of peak warming, leading to an unavoidable transgression of 1.5 °C in all models, and 2 °C in some. In contrast to the incremental changes as foreseen by current plans, ambitious peak warming targets require decisive emission cuts until 2030, with the most substantial contribution to decarbonization coming from the power sector. Therefore, investments into low-carbon power generation need to increase beyond current levels to meet the Paris goals, especially for solar and wind technologies and related system enhancements for electricity transmission, distribution and storage. Estimates on absolute investment levels, up-scaling of other low-carbon power generation technologies and investment shares in less ambitious scenarios vary considerably across models. In scenarios limiting peak warming to below 2 °C, while coal is phased out quickly, oil and gas are still being used significantly until 2030, albeit at lower than current levels. This requires continued investments into existing oil and gas infrastructure, but investments into new fields in such scenarios might not be needed. The results show that credible and effective policy action is essential for ensuring efficient allocation of investments aligned with medium-term climate targets.
