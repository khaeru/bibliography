The estimation of Sub-national purchasing power parities (PPPs) for countries where the regions and provinces have different level of development is fundamental for income, consumption, standard of living real term comparisons, as well as for measuring cross-region welfare inequality. This is even truer for large countries like China, where the above aggregates exhibit great variability among provinces. The aim of this paper is to compute the price level differences, measured by the PPPs, for 31 Chinese Provinces and Municipal Cities, based on a sample of 62 goods and services for the year 2014. To our knowledge, this is the first attempt to do it since many years. After a short review of previous studies on China cross-province and municipal cities price level differences measurement and the illustration of methodology and data used, the results of our elaborations are presented and discussed. Taking Beijing as the base area, there is evidence that the PPP max/min ratio is 1.74, confirming the common belief that China cross-province and municipal cities price levels are significantly different.
