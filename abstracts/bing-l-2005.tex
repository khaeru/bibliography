Public–private partnerships (PPPs) are increasingly used in the United Kingdom's public facilities and services provision through the Private Finance Initiative (PFI). Despite some casualties, PPP/PFI projects have been undertaken successfully, but the reasons for success are not entirely clear. Questionnaire survey research examined the relative importance of 18 potential critical success
factors (CSF) for PPP/PFI construction projects in the UK. The results show that the three most important factors are: ‘a strong and good private consortium’, ‘appropriate risk allocation’ and ‘available
financial market’. Factor analysis revealed that appropriate factor groupings for the 18 CSFs are: effective procurement, project implementability, government guarantee, favourable economic conditions and
available financial market. These findings should influence policy development towards PPPs and the manner in which partners go about the development of PFI projects. [ABSTRACT FROM AUTHOR]