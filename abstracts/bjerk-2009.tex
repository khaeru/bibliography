Researchers are often interested in estimating the causal effect of some treatment on individual criminality. For example, two recent relatively prominent papers have attempted to estimate the respective direct effects of marriage and gang participation on individual criminal activity. One difficulty to overcome is that the treatment is often largely the product of individual choice. This issue can cloud causal interpretations of correlations between the treatment and criminality since those choosing the treatment (e.g. marriage or gang membership) may have differed in their criminality from those who did not even in the absence of the treatment. To overcome this potential for selection bias researchers have often used various forms of individual fixed-effects estimators. While such fixed-effects estimators may be an improvement on basic cross-sectional methods, they are still quite limited when it comes to uncovering a true causal effect of the treatment on individual criminality because they may fail to account for the possibility of dynamic selection. Using data from the NSLY97, I show that such dynamic selection can potentially be quite large when it comes to criminality, and may even be exacerbated when using more advanced fixed-effects methods such as Inverse Probability of Treatment Weighting (IPTW). Therefore substantial care must be taken when it comes to interpreting the results arising from fixed-effects methods.