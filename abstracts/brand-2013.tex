The transition to a low carbon transport world requires a host of demand and supply policies to be developed and deployed. Pricing and taxation of vehicle ownership plays a major role, as it affects purchasing behavior, overall ownership and use of vehicles. There is a lack in robust assessments of the life cycle energy and environmental effects of a number of key car pricing and taxation instruments, including graded purchase taxes, vehicle excise duties and vehicle scrappage incentives. This paper aims to fill this gap by exploring which type of vehicle taxation accelerates fuel, technology and purchasing behavioral transitions the fastest with (i) most tailpipe and life cycle greenhouse gas emissions savings, (ii) potential revenue neutrality for the Treasury and (iii) no adverse effects on car ownership and use. The \{UK\} Transport Carbon Model was developed further and used to assess long term scenarios of low carbon fiscal policies and their effects on transport demand, vehicle stock evolution, life cycle greenhouse gas emissions in the UK. The modeling results suggest that policy choice, design and timing can play crucial roles in meeting multiple policy goals. Both \{CO2\} grading and tightening of \{CO2\} limits over time are crucial in achieving the transition to low carbon mobility. Of the policy scenarios investigated here the more ambitious and complex car purchase tax and feebate policies are most effective in accelerating low carbon technology uptake, reducing life cycle greenhouse gas emissions and, if designed carefully, can avoid overburdening consumers with ever more taxation whilst ensuring revenue neutrality. Highly graduated road taxes (or VED) can also be successful in reducing emissions; but while they can provide handy revenue streams to governments that could be recycled in accompanying low carbon measures they are likely to face opposition by the driving population and car lobby groups. Scrappage schemes are found to save little carbon and may even increase emissions on a life cycle basis. The main policy implication of this work is that in order to reduce both direct and indirect greenhouse gas emissions from transport governments should focus on designing incentive schemes with strong up-front price signals that reward ‘low carbon’ and penalize ‘high carbon’. Policy instruments should also be subject to early scrutiny of the longer term impacts on government revenue and pay attention to the need for flanking policies to boost these revenues and maintain the marginal cost of driving.