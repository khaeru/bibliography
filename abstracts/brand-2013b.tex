Carbon dioxide (CO2) emissions from motorised travel are hypothesised to be associated with individual, household, spatial and other environmental factors. Little robust evidence exists on who contributes most (and least) to travel \{CO2\} and, in particular, the factors influencing commuting, business, shopping and social travel CO2. This paper examines whether and how demographic, socio-economic and other personal and environmental characteristics are associated with land-based passenger transport and associated \{CO2\} emissions. Primary data were collected from 3474 adults using a newly developed survey instrument in the iConnect study in the UK. The participants reported their past-week travel activity and vehicle characteristics from which \{CO2\} emissions were derived using an adapted travel emissions profiling method. Multivariable linear and logistic regression analyses were used to examine what characteristics predicted higher \{CO2\} emissions. \{CO2\} emissions from motorised travel were distributed highly unequally, with the top fifth of participants producing more than two fifth of emissions. Car travel dominated overall \{CO2\} emissions, making up 90\% of the total. The strongest independent predictors of \{CO2\} emissions were owning at least one car, being in full-time employment and having a home-work distance of more than 10 km. Income, education and tenure were also strong univariable predictors of \{CO2\} emissions, but seemed to be further back on the causal pathway than having a car. Male gender, late-middle age, living in a rural area and having access to a bicycle also showed significant but weaker associations with emissions production. The findings may help inform the development of climate change mitigation policies for the transport sector. Targeting individuals and households with high car ownership, focussing on providing viable alternatives to commuting by car, and supporting planning and other policies that reduce commuting distances may provide an equitable and efficient approach to meeting carbon mitigation targets.