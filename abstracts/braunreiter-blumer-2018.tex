Scenarios are a key instrument to guide decision-making in the face of an uncertain future. In the field of energy, scenarios are often published to inform external stakeholders who are not part of the scenario development. This study explores how researchers, a key stakeholder group in shaping the energy future, use energy scenarios. It analyses the case of Switzerland, where several competing scenarios have been developed in reaction to the governmental decision to phase-out nuclear power. 16 structured in-depth interviews with researchers from different disciplinary backgrounds were conducted. While most interviewees use public energy scenarios, there are two contrasting user types. The first group, labelled divers, primarily uses scenarios as a data source, whereas the other group, the sailors, refers to them as plausible energy futures. We identified different interpretations of scenario content between sailors and divers, which result from the quantitative modelling on which contemporary energy scenarios are based. Due to a lack of guidance from modellers and missing qualitative information, energy scenarios are prone to misconceptions and distortions in their interpretation by external users.