 Throughout the world, urban areas have been rapidly expanding, exacerbating the problem of many public transport (PT) operators providing service over different governmental jurisdictions. Over the past five decades, Germany, Austria, and Switzerland have successfully implemented regional PT associations (called Verkehrsverbund or VV), which integrate services, fares, and ticketing while coordinating public transport planning, marketing, and customer information throughout metropolitan areas, and in some cases, entire states. A key difference between VVs and other forms of regional PT coordination is the collaboration and mutual consultation of government jurisdictions and PT providers in all decision-making. This article examines the origins of VVs, their spread to 13 German, Austrian, and Swiss metropolitan areas from 1967 to 1990, and their subsequent spread to 58 additional metropolitan areas from 1991 to 2017, now serving 85\% of Germany's and 100\% of Austria's population. The VV model has spread quickly because it is adaptable to the different degrees and types of integration needed in different situations. Most of the article focuses on six case studies of the largest VVs: Hamburg (opened in 1967), Munich (1971), Rhine-Ruhr (1980), Vienna (1984), Zurich (1990), and Berlin-Brandenburg (1999). Since 1990, all six of those VVs have increased the quality and quantity of service, attracted more passengers, and reduced the percentage of costs covered by subsidies. By improving PT throughout metropolitan areas, VVs provide an attractive alternative to the private car, helping to explain why the car mode share of trips has fallen since 1990 in all of the case studies. 
