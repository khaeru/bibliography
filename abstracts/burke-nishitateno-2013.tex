Countries differ considerably in terms of the price drivers pay for gasoline. This paper uses data for 132 countries for the period 1995–2008 to investigate the implications of these differences for the consumption of gasoline for road transport. To address the potential for simultaneity bias, we use both a country's oil reserves and the international crude oil price as instruments for a country's average gasoline pump price. We obtain estimates of the long-run price elasticity of gasoline demand of between −0.2 and −0.5. Using newly available data for a sub-sample of 43 countries, we also find that higher gasoline prices induce consumers to substitute to vehicles that are more fuel-efficient, with an estimated elasticity of +0.2. Despite the small size of our elasticity estimates, there is considerable scope for low-price countries to achieve gasoline savings and vehicle fuel economy improvements via reducing gasoline subsidies and/or increasing gasoline taxes.