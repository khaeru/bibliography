One day in Beijing provides a jarring snapshot of motorization issues in China. Beijing is considered the most motorized city in China, and the consequent air pollution and congestion are stark. However, despite worsening conditions and rising prices, owning a car is often portrayed as a natural expectation, or even requirement, for rising middle class Chinese. Prior studies suggest that the desire for cars is a values-based perception, influenced by desires for social status and materialistic aspirations, rather than an instrumental desire. Through semi-structured interviews, this study explores the life aspirations and values of post-80’s generation white collar workers, and how important car ownership is to them. While all interviewees express desire to own a car at some point, the motivations for doing so were quite different. Men felt a significant pressure from women and society to ‘provide’, which includes having a car. Women all saw having a car as necessary, but not because of prestige or status. Overall, the assumption by these rising middle class Beijingers is that owning a car is an expectation rather than a luxury. Strong value associations with driving already exist, independent of driving experience, suggesting the role of advertising and peer conformity. Almost no ‘rational thinking’ weighing mobility options occurred, but thinking was highly emotionally coded.
