This paper presents a study of spatio-temporal patterns and city-level determinants of private car ownership in urban China. Car ownership data from 1990 to 2009 were collected from 235 cities of the prefecture-level and above. Panel data models are developed to examine the effects of city's socioeconomic, spatial and transport variables on private car ownership development. We find that car ownership development in Chinese cities has experienced stages of steady and rapid growths in the years before and after 2005; cities of larger sizes as well as in the coastal region have a higher level of car ownership and have experienced much faster growth in the entire period; and the most important determinants of car ownership are economic development related, including gross domestic product per capita and disposable income per capita, which both have positive and significant effects on car ownership in different periods and for different sizes of cities. This study adds to the literature some empirical evidence on the spatio-temporal patterns of private car ownership in urban China and the importance of socioeconomic, spatial and transport-related factors in determining the level of car ownership of cities.
