Based on econometric estimation using data from the Chinese Urban Household Survey, we develop a preferred forecast range of 85–143 percent growth in residential per capita electricity demand over 2009–2025. Our analysis suggests that per capita income growth drives a 43% increase, with the remainder due to an unexplained time trend. Roughly one-third of the income-driven demand comes from increases in the stock of specific major appliances, particularly AC units. The other two-thirds comes from non-specific sources of income-driven growth and is based on an estimated income elasticity that falls from 0.28 to 0.11 as income rises. While the stock of refrigerators is not projected to increase, we find that they contribute nearly 20 percent of household electricity demand. Alternative plausible time trend assumptions are responsible for the wide range of 85–143 percent. Meanwhile we estimate a price elasticity of demand of −0.7. These estimates point to carbon pricing and appliance efficiency policies that could substantially reduce demand.