Numerous studies have found that suburban residents drive more and walk less than residents in traditional neighbourhoods. What is less well understood is the extent to which the observed patterns of travel behaviour can be attributed to the residential built environment (BE) itself, as opposed to attitude‐induced residential self‐selection. To date, most studies addressing this self‐selection issue fall into nine methodological categories: direct questioning, statistical control, instrumental variables, sample selection, propensity score, joint discrete choice models, structural equations models, mutually dependent discrete choice models and longitudinal designs. This paper reviews 38 empirical studies using these approaches. Virtually all of the studies reviewed found a statistically significant influence of the BE remaining after self‐selection was accounted for. However, the practical importance of that influence was seldom assessed. Although time and resource limitations are recognized, we recommend usage of longitudinal structural equations modelling with control groups, a design which is strong with respect to all causality requisites. 
