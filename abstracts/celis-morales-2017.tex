Objective~To investigate the association between active commuting and incident cardiovascular disease (CVD), cancer, and all cause mortality.Design~Prospective population based study. Setting~UK Biobank.Participants~263 450 participants (106 674 (52\%) women; mean age 52.6), recruited from 22 sites across the UK. The exposure variable was the mode of transport used (walking, cycling, mixed mode v non-active (car or public transport)) to commute to and from work on a typical day.Main outcome measures~Incident (fatal and non-fatal) CVD and cancer, and deaths from CVD, cancer, or any causes.Results~2430 participants died (496 were related to CVD and 1126 to cancer) over a median of 5.0 years (interquartile range 4.3-5.5) follow-up. There were 3748 cancer and 1110 CVD events. In maximally adjusted models, commuting by cycle and by mixed mode including cycling were associated with lower risk of all cause mortality (cycling hazard ratio 0.59, 95\% confidence interval 0.42 to 0.83, P=0.002; mixed mode cycling 0.76, 0.58 to 1.00, P\&lt;0.05), cancer incidence (cycling 0.55, 0.44 to 0.69, P\&lt;0.001; mixed mode cycling 0.64, 0.45 to 0.91, P=0.01), and cancer mortality (cycling 0.60, 0.40 to 0.90, P=0.01; mixed mode cycling 0.68, 0.57 to 0.81, P\&lt;0.001). Commuting by cycling and walking were associated with a lower risk of CVD incidence (cycling 0.54, 0.33 to 0.88, P=0.01; walking 0.73, 0.54 to 0.99, P=0.04) and CVD mortality (cycling 0.48, 0.25 to 0.92, P=0.03; walking 0.64, 0.45 to 0.91, P=0.01). No statistically significant associations were observed for walking commuting and all cause mortality or cancer outcomes. Mixed mode commuting including walking was not noticeably associated with any of the measured outcomes.Conclusions~Cycle commuting was associated with a lower risk of CVD, cancer, and all cause mortality. Walking commuting was associated with a lower risk of CVD independent of major measured confounding factors. Initiatives to encourage and support active commuting could reduce risk of death and the burden of important chronic conditions.