The suburbanization of large Chinese cities has placed many residents in locations that are far less accessible than their prior residences, requiring motorized travel. This paper examines the impacts of relocation to outlying areas on job accessibility, commuting mode choice, and commuting durations based on a current-day and retrospective survey of recent movers to three suburban neighborhoods in Shanghai. Job accessibility levels were found to decline dramatically following the move, matched by increased motorized travel and longer commute durations. Relocating to a suburban area near a metrorail station, however, was found to moderate losses in job accessibility and for many, encourage switches from non-motorized to transit commuting. We conclude that transit-oriented development holds considerable promise for placing rapidly suburbanizing Chinese cities on a more sustainable pathway.
