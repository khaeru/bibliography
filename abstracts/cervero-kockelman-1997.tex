The built environment is thought to influence travel demand along three principal dimensions —density, diversity, and design. This paper tests this proposition by examining how the ‘3Ds’ affect trip rates and mode choice of residents in the San Francisco Bay Area. Using 1990 travel diary data and land-use records obtained from the U.S. census, regional inventories, and field surveys, models are estimated that relate features of the built environment to variations in vehicle miles traveled per household and mode choice, mainly for non-work trips. Factor analysis is used to linearly combine variables into the density and design dimensions of the built environment. The research finds that density, land-use diversity, and pedestrian-oriented designs generally reduce trip rates and encourage non-auto travel in statistically significant ways, though their influences appear to be fairly marginal. Elasticities between variables and factors that capture the 3Ds and various measures of travel demand are generally in the 0.06 to 0.18 range, expressed in absolute terms. Compact development was found to exert the strongest influence on personal business trips. Within-neighborhood retail shops, on the other hand, were most strongly associated with mode choice for work trips. And while a factor capturing ‘walking quality’ was only moderately related to mode choice for non-work trips, those living in neighborhoods with grid-iron street designs and restricted commercial parking were nonetheless found to average significantly less vehicle miles of travel and rely less on single-occupant vehicles for non-work trips. Overall, this research shows that the elasticities between each dimension of the built environment and travel demand are modest to moderate, though certainly not inconsequential. Thus it supports the contention of new urbanists and others that creating more compact, diverse, and pedestrian-orientated neighborhoods, in combination, can meaningfully influence how Americans travel.