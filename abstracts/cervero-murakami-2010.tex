 Concerns over rising fuel prices and greenhouse-gas emissions have prompted research into the influences of built environments on travel, notably vehicle miles traveled (VMT). On the basis of data from 370 US urbanized areas and using structural equation modeling, population densities are shown to be strongly and positively associated with VMT per capita (direct effect elasticity = ⊟0.604); however, this effect is moderated by the traffic-inducing effects of denser urban settings having denser road networks and better local-retail accessibility (indirect effect elasticity = 0.223, yielding a net effect elasticity = ⊟0.381). Accessibility to basic employment has comparatively modest effects, as do size of urbanized area, and rail-transit supplies and usage. Nevertheless, urban planning and city design should be part of any strategic effort to shrink the environmental footprint of the urban transportation sector. 