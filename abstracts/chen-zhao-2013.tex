Increased automobile ownership and use in China over the last two decades has increased energy consumption, worsened air pollution, and exacerbated congestion. However, the countrywide growth in car ownership conceals great variation among cities. For example, Shanghai and Beijing each had about 2 million motor vehicles in 2004, but by 2010, Beijing had 4.8 million motor vehicles whereas Shanghai had only 3.1 million. Among the factors contributing to this divergence is Shanghai’s vehicle control policy, which uses monthly license auctions to limit the number of new cars. The policy appears to be effective: in addition to dampening growth in car ownership, it generates annual revenues up to 5 billion CNY (800 million USD). But, despite these apparent successes, the degree to which the public accepts this policy is unknown. This study surveys 524 employees at nine Shanghai companies to investigate the policy acceptance of Shanghai’s license auction by the working population, and the factors that contribute to that acceptance: Perceived policy effectiveness, affordability, equity concerns, and implementation. Respondents perceive the policy to be effective, but are moderately negative towards the policy nonetheless. However, they expect that others accept the policy more than they do. Respondents also hold consistently negative perceptions about the affordability of the license, the effects on equity, and the implementation process. Revenue usage is not seen as transparent, which is exacerbated by a perception that government vehicles enjoy advantages in obtaining a license, issues with the bidding process and technology, and difficulties in obtaining information about the auction policy. Nevertheless, respondents believe that license auctions and congestion charges are more effective and acceptable than parking charges and fuel taxes. To improve public acceptability of the policy, we make five recommendations: Transparency in revenue usage; transparency in government vehicle licensing and use, categorising licenses by vehicle type, implementation and technology improvements to increase bidding convenience, and policies that restrict vehicle usage in congested locations.
