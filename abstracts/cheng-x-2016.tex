Compared with conventional vehicles, electric vehicles (EVs) offer the benefits of replacing petroleum consumption and reducing air pollutions. However, there have been controversies over greenhouse gas (GHG) emissions of EVs from the life-cycle perspective in China’s coal-dominated power generation context. Besides, it is in doubt whether the cost-effectiveness of EVs in China exceeds other fuel-efficient vehicles considering the high prices. In this study, we compared the life-cycle GHG emissions of existing vehicle models in the market. Afterwards, a cost model is established to compare the total costs of vehicles. Finally, the cost-effectiveness of different vehicle types are compared. It is concluded that the GHG emission intensity of EVs is lower than reference and hybrid vehicles currently and is expected to decrease with the improvement of the power grid. The total cost of EVs is relatively high compared with reference gasoline vehicles in 2014 but it is expected that EVs will possess an improved cost-competitiveness in the future. In terms of cost-effectiveness, medium-range EVs do not have an obvious advantage over other fuel-efficient vehicles currently. But the cost-effectiveness of EVs is predicted to become better in the next ten years.
