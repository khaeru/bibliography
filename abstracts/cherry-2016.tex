The rapid adoption of electric bikes (e-bikes) (~150 million in 10 years) has come with debate over their role in China's urban transportation system. While there has been some research quantifying impacts of e-bikes on the transportation system, there has been little work tracking e-bike use patterns over time. This paper investigates e-bike use over a 6-year period. Four bi-annual travel diary surveys of e-bike users were conducted between 2006 and 2012 in Kunming, China. Choice models were developed to investigate factors influencing mode-transition and motorization pathways. As expected, income and vehicle ownership strongly influence car-based transitions. Younger and female respondents were more likely to choose car-based modes. Systematic and unobserved changes over time (time-dynamics) favor car-based modes, with the exception of previous car users who already shifted away from cars being less likely to revert to cars over time. E-bikes act as an intermediate mode, interrupting the transition from bicycle to bus and from bus to car. Over 6 years, e-bikes are displacing prospective bus (65→55\%), car/taxi (15→24\%) and bicycle (19→7\%) trips. Over 40\% of e-bike riders now have household car access so e-bikes are effectively replacing many urban car trips.
