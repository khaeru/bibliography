 The economic reforms in China since 1979 and consequent increases in disposable income have caused total gasoline consumption to soar nearly 240\% between 1980 and 1999. As the growth rate of gasoline consumption is expected to be high due to the increased economic activity resulting from China's re-accession to the WTO, the government must understand the implications for economic growth and balance of payments. Using cointegration techniques, it was found that, between 1980 and 1999, demand for gasoline was relatively inelastic to price changes, both in the short and long terms. The long-run income elasticity was 0.97, implying that the future growth rate of gasoline consumption will be close to the growth rate of the economy, which is predicted to be about 7\% per annum from 2001 to 2005, and 5-6\% over the decade thereafter. 