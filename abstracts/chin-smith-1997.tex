Rapid growth in real income in many countries of South-East Asia has led to large increases in the ownership and usage of automobiles. In many major cities this has resulted in chronic traffic congestion. Singapore has so far avoided the worst excesses of this problem, by a series of policy measures aimed at controlling automobile ownership as well as usage. In the latest moves (from 1990), a quantity rationing system has been introduced to impose close control on the number of additional automobiles allowed in Singapore, augmenting a battery of price-based policies introduced over the previous 15 yr. This paper examines the theoretical basis for this switch in the focus of policy, and presents an econometric investigation intended to evaluate the overall success of the programme in controlling the automobile population.
