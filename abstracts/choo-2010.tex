With aggregate data from the U.S. Consumer Expenditure Survey for 19 years, 1984 through 2002, this study analyzes relationships between expenditures on transportation and communications. Several classification schemes for expenditure categories were used, from the most aggregate [two categories (transportation and communications)] to the most disaggregate [nine transportation categories (new vehicle purchases, used vehicle purchases, vehicle finance charges, gasoline and motor oil, vehicle maintenance and repairs, vehicle insurance, public transportation, out-of-town lodging, and other entertainment including bikes and recreational vehicles) and five communications categories (telephone service; miscellaneous household equipment including phones and computers; television, radio, and sound equipment; postage and stationery; and reading)]. Aggregate demand system modeling (in particular, the linear approximate almost ideal demand system) was then used to determine the relationships between expenditures on transportation and those on communications, again for several different classifications. The model results indicate that transportation and communications categories have substitution and complementarity relationships, often not symmetric. However, a dominant effect of complementarity can be found in the influence of communications on transportation.