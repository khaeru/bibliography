The claim by China experts that the Chinese state lacks the capability to practice pro-active industrial policy throws China's future into doubt. This article argues the contrary by examining the development of the Chinese automobile industry and the evolution of its industrial policy since 1978. The central state's policy may have been ineffective at first, but continued to improve, layer-by-layer, by taking into account results of local experiments, and being propelled by a strong catch-up consensus providing performance standards to establish national industries. China's learning process thus renders its industrial policy model effective in the long term.
