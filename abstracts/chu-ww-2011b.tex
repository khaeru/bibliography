Whether and how will the state treat different kinds of enterprises, state-owned enterprises (SOEs) and non-SOEs, in the related industry differently? Does this concern give its industrial policy any special Chinese characteristics? This article looks at a particular aspect of the Chinese automotive industry policy, that is, regulating entries, which poses special problems for the government. It explores why the government still retains this method of control even after it has been shown to be ineffective, and how the government tries to reconcile it with the aim of promoting entrepreneurship. The government finds that it cannot do away with entrepreneurship brought by the unplanned entrants (SOEs or not) to keep the industry competitive. Moreover, with SOEs accounting for the major part of the auto industry, the government has to protect the SOEs and propel them to upgrade at the same time. The government is thus likely to continue regulating entries, while trying to find a balance between the needs of keeping entrepreneurship and managing SOEs.
