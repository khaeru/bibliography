1. The theta-logistic is a simple and flexible model for describing how the growth rate of a population slows as abundance increases. Starting at rm (taken as the maximum population growth rate), the growth response decreases in a convex or concave way (according to the shape parameter θ) to zero when the population reaches carrying capacity.2. We demonstrate that fitting this model to census data is not robust and explain why. The parameters θ and rm are able to play-off against each other (providing a constant product), thus allowing both to adopt extreme and ecologically implausible values.3. We use simulated data to examine: (i) a population fluctuating around a constant carrying capacity (K); (ii) recovery of a population from 10\% of carrying capacity; and (iii) a population subject to variation in K. We show that estimates of extinction risk depending on this or similar models are therefore prone to imprecision. We refute the claim that concave growth responses are shown to dominate in nature.4. As the model can also be sensitive to temporal variation in carrying capacity, we argue that the assumption of a constant carrying capacity is both problematic and presents a fruitful direction for the development of phenomenological density-feedback models.