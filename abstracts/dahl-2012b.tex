Price and income elasticities of transport fuel demand have numerous applications. They help forecast increases in fuel consumption as countries get richer, they help develop appropriate tax policies to curtail consumption, help determine how the transport fuel mix might evolve, and show the price response to a fuel disruption. Given their usefulness, it is understandable why hundreds of studies have focused on measuring such elasticities for gasoline and diesel fuel consumption. In this paper, I focus my attention on price and income elasticities in the existing studies to see what can be learned from them. I summarize the elasticities from these historical studies. I use statistical analysis to investigate whether income and price elasticities seem to be constant across countries with different incomes and prices. Although income and price elasticities for gasoline and diesel fuel are not found to be the same at high and low incomes and at high and low prices, patterns emerge that allow me to develop suggested price and income elasticities for gasoline and diesel demand for over one hundred countries. I adjust these elasticities for recent fuel mix policies, and suggest an agenda of future research topics.