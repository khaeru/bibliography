This paper explores how China’s household consumption patterns over the period 2005–2050 influence the total energy demand and carbon dioxide CO$_2$ emissions in two baseline scenarios, and how it influences carbon prices as well as the economic cost in the corresponding carbon mitigation scenarios. To this end we first put forward two possible household consumption expenditure patterns up to 2050 using the Working–Leser model, taking into account total expenditure increase and urbanization. For comparison, both expenditure patterns are then incorporated in a hybrid recursive dynamic computable general equilibrium model. The results reveal that as income level increases in the coming decades, the direct and indirect household energy requirements and CO$_2$ emissions would rise drastically. When household expenditure shifts from material products and transport to service-oriented goods, around 21,000 mtce of primary energy and 45 billion tons of CO$_2$ emissions would be saved over the 45-year period from 2005 to 2050. Moreover, carbon prices in the dematerialized mitigation scenario would fall by 13\% in 2050, thus reducing the economic cost.
