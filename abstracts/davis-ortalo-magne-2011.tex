New evidence from the 1980, 1990, and 2000 Decennial Census of Housing indicates that expenditure shares on housing are constant over time and across \{US\} metropolitan statistical areas (MSA). Consistent with this observation, we consider a model in which identical households with Cobb–Douglas preferences for housing and non-housing consumption choose a location and locations differ with respect to income earned by their residents. The model predicts that the relative price of housing of any two \{MSAs\} disproportionately reflects differences in incomes of those \{MSAs\} and is independent of housing supply in each MSA. According to the predictions of our calibrated model, the dispersion of rental prices across low- and high-wage \{MSAs\} should be larger than we observe: High-wage \{MSAs\} like San Francisco are puzzlingly inexpensive relative to low-wage \{MSAs\} like Pittsburgh.