What distinguishes the Chinese practice of transferring policy ideas and institutions from examples observed elsewhere in the world can be described in two words: gradualism and eclecticism. In contradistinction to other (Post) Communist countries, actors operating in the Chinese political and socio-economic systems were not so taken aback by developments in 1989 that these completed collapsed. Nor were they overhauled in rigorous ways so as to realize a brand new start in which Communist and authoritarian remnants of the past were to be completely effaced. Rather did policy makers keenly observe developments and spot promising examples elsewhere in the world to draw lessons from. These were then reassembled onto existing institutional frameworks. In this article, it is claimed that this cautious and selective approach reflects a more generic Chinese tradition of institutional bricolage. This tradition of cobbling together various foreign and domestic policy ideas in modular fashion is illustrated with the modern day example of eco city development in China.
