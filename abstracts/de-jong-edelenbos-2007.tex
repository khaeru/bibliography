Much of the recent academic literature on spatial planning in Europe focuses on either cross-national comparison of planning frameworks and planning practices or on transnational and transregional initiatives and their impact on planning in European countries. From those publications, it can be gleaned how similar themes are translated differentially in different national contexts. Although it is also a great source of European integration and harmonization, the phenomenon of the knowledge exchange within transnational expert networks of European planners at the level of cities has received less attention. In this paper, the knowledge exchange among planners in such a network is studied, highlighting the role of “transfer agents” (academic and/or policy experts operating in communities in different policy arenas) in the exchange process. It builds on the insights from existing literature on policy transfer and policy learning, and tries to add a new perspective on this body of literature from an insiders' perspective, i.e. participatory observation. The idea is that policy transfer can be fruitfully approached as a process of knowledge and information transfer between producers, senders, facilitators and recipients. Often this exchange is to a very large extent a process of absorbing appealing labels for policy solutions from the international or national policy levels, and then adopting an interpretation of it suitable to one's own context. The authors try to give meaning to this exchange process by using two mechanisms, i.e. social interaction and conceptual replication. By combining these two mechanisms the authors try to uncover which policy lessons are being transferred among seven European cities that joined the expert network on European sustainable urban development (Pegasus).
