MONASH models are descended from Johansen's 1960 model of Norway. The first MONASH model was ORANI, used in Australia's tariff debate of the 1970s. Johansen's influence combined with institutional arrangements in their development gave MONASH models distinctive characteristics, facilitating a broad range of policy-relevant applications. MONASH models currently operate in numerous countries to provide insights on a variety of questions including:the effects on:macro, industry, regional, labor market, distributional and environmental variablesof changes in:taxes, public consumption, environmental policies, technologies, commodity prices, interest rates, wage-setting arrangements, infrastructure and major-project expenditures, and known levels and exploitability of mineral deposits (the Dutch disease). MONASH models are also used for explaining periods of history, estimating changes in technologies and preferences and generating baseline forecasts. Creation of MONASH models involved a series of enhancements to Johansen's model, including: (i) a computational procedure that eliminated Johansen's linearization errors without sacrificing simplicity; (ii) endogenization of trade flows by introducing into computable general equilibrium (CGE) modeling imperfect substitution between imported and domestic varieties (the Armington assumption); (iii) increased dimensionality allowing for policy-relevant detail such as transport margins; (iv) flexible closures; and (v) complex functional forms to specify production technologies. As well as broad theoretical issues, this chapter covers data preparation and introduces the GEMPACK purpose-built CGE software. MONASH modelers have responded to client demands by developing four modes of analysis: historical, decomposition, forecast and policy. Historical simulations produce up-to-date data, and estimate trends in technologies, preferences and other naturally exogenous but unobservable variables. Decomposition simulations explain historical episodes and place policy effects in historical context. Forecast simulations provide baselines using extrapolated trends from historical simulations together with specialist forecasts. Policy simulations generate effects of policies as deviations from baselines. To emphasize the practical orientation of MONASH models, the chapter starts with a MONASH-style policy story.