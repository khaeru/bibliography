Energy demand of a transport sector has constantly been increasing in the recent years, consuming one third of the total final energy demand in the European Union (EU) over the last decade. A transition of this sector towards sustainable one is facing many challenges in terms of suitable technology and energy resources. Especially challenging transition is envisaged for heavy-weight, long-range vehicles and airplanes. A detailed literature review was carried out in order to detect the current state of the research on clean transport sector, as well as to point out the gaps in the research. In order to calculate the resources needed for the transition towards completely renewable transport sector, four main alternatives to the current fossil fuel systems were assessed and their potential was quantified, i.e. biofuels, hydrogen, synthetic fuels (electrofuels) and electricity. Results showed that electric modes of transport have the largest benefits and should be the main aim of the transport transition. It was calculated that 72.3% of the transport energy demand on the EU level could be directly electrified by the technology existing today. For the remaining part of the transport sector a significant demand for energy resources exists, i.e. 3069TWh of additional biomass was needed in the case of biofuels utilization scenario while 2775TWh of electricity and 925TWh of heat were needed in the case of renewable electrofuels produced using solid oxide electrolysis scenario.