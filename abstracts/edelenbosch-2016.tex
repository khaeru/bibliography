The transport sector is growing fast in terms of energy use and accompanying greenhouse gas emissions. Integrated assessment models (IAMs) are used widely to analyze energy system transitions over a decadal time frame to help inform and evaluating international climate policy. As part of this, \{IAMs\} also explore pathways of decarbonizing the transport sector. This study quantifies the contribution of changes in activity growth, modal structure, energy intensity and fuel mix to the projected passenger transport carbon emission pathways. The Laspeyres index decomposition method is used to compare results across models and scenarios, and against historical transport trends. Broadly-speaking the models show similar trends, projecting continuous transport activity growth, reduced energy intensity and in some cases modal shift to carbon-intensive modes - similar to those observed historically in a business-as-usual scenario. In policy-induced mitigation scenarios further enhancements of energy efficiency and fuel switching is seen, showing a clear break with historical trends. Reduced activity growth and modal shift (towards less carbon-intensive modes) only have a limited contribution to emission reduction. Measures that could induce such changes could possibly complement the aggressive, technology switch required in the current scenarios to reach internationally agreed climate targets.