Income and fuel price pathways are key determinants in projections of the energy system in integrated assessment models. In recent years, more details have been added to the transport sector representation in these models. To better understand the model dynamics, this manuscript analyses transport fuel demand elasticities to projected income and fuel price levels. Fuel price shocks were simulated under various scenario assumptions to isolate price effects on energy demand and create a transparent environment to compare fuel demand response. Interestingly, the models show very comparable oil price elasticity values for the projected first 10–20 years that are also close to the range described in the empirical literature. When looking at the very long term (30–40 years), demand elasticity values widely vary between models, between 0.4 and −1.9, showing either continuous demand or increased demand responses over time. The latter can be the result of long response time to fuel price shocks, availability of new technologies, and feedback effects on fuel prices. The projected transport service demand is more responsive to changes in income than fuel price pathways, corresponding with the literature. Calculating the models' inherent elasticities proved to be a suitable method to evaluate model behaviour and its application is also recommended for other models as well as other sectors represented in integrated assessment models.