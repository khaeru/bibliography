This paper discusses a specific part of sustainable transport policy, namely policies to reduce greenhouse gas emissions from the transport sector. We explain how assessments of such policies will overestimate their effectiveness if market responses are not taken into account. The substantial difference between market price and extraction cost of oil means that consumption reductions will be watered down by price responses causing increased consumption in other places (spatial leakage) and in the future (intertemporal leakage). The difference between market price and extraction cost also has negative implications for the viability of alternative technologies. Leakage effects become larger when consumption reductions are only undertaken by a subset of countries: we review some theoretical evidence why strong binding international climate agreements are so difficult to reach and to enforce. All this may require rethinking climate policies for the transport sector: What policies remain cost effective for reducing greenhouse gas emissions?