New spatial media – the informational artefacts and mediating technologies of the geoweb – represent new opportunities for activist, civic, grassroots, indigenous and other groups to leverage web-based geographic information technologies in their efforts to effect social change. Drawing upon evidence from an inductive analysis of five online initiatives that engage new spatial media in activism and civic engagement, we explore new dimensions of the knowledge politics advanced through new spatial media and the mechanisms through which they emerge. ‘Knowledge politics’ refers to the use of particular information content, forms of representation or ways of analysing and manipulating information to try to establish the authority or legitimacy of knowledge claims. The five new spatial media initiatives we analyse here introduce new dimensions to the modes of collecting, validating and representing information, when considered against practices of many activist/civic encounters with other kinds of geographic information technologies, such as GIS. The significance of these practices is not in their (arguable) newness, but rather their role in advancing different epistemological strategies for establishing the legitimacy and authority of knowledge claims. Specifically, these new knowledge politics entail a deployment of geovisual artefacts to structure a visual experience; a prioritisation of individualised interactive/exploratory ways of knowing; hyper-granular, highly immediate, experiential cartographic representations de-coupled from conventional practices of cartographic abstraction; and approaches to asserting credibility through witnessing, peer verification and transparency.