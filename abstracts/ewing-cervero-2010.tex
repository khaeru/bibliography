 Problem: Localities and states are turning to land planning and urban design for help in reducing automobile use and related social and environmental costs. The effects of such strategies on travel demand have not been generalized in recent years from the multitude of available studies. Purpose: We conducted a meta-analysis of the built environment-travel literature existing at the end of 2009 in order to draw generalizable conclusions for practice. We aimed to quantify effect sizes, update earlier work, include additional outcome measures, and address the methodological issue of self-selection. Methods: We computed elasticities for individual studies and pooled them to produce weighted averages. Results and conclusions: Travel variables are generally inelastic with respect to change in measures of the built environment. Of the environmental variables considered here, none has a weighted average travel elasticity of absolute magnitude greater than 0.39, and most are much less. Still, the combined effect of several such variables on travel could be quite large. Consistent with prior work, we find that vehicle miles traveled (VMT) is most strongly related to measures of accessibility to destinations and secondarily to street network design variables. Walking is most strongly related to measures of land use diversity, intersection density, and the number of destinations within walking distance. Bus and train use are equally related to proximity to transit and street network design variables, with land use diversity a secondary factor. Surprisingly, we find population and job densities to be only weakly associated with travel behavior once these other variables are controlled. Takeaway for practice: The elasticities we derived in this meta-analysis may be used to adjust outputs of travel or activity models that are otherwise insensitive to variation in the built environment, or be used in sketch planning applications ranging from climate action plans to health impact assessments. However, because sample sizes are small, and very few studies control for residential preferences and attitudes, we cannot say that planners should generalize broadly from our results. While these elasticities are as accurate as currently possible, they should be understood to contain unknown error and have unknown confidence intervals. They provide a base, and as more built-environment/travel studies appear in the planning literature, these elasticities should be updated and refined. Research support: U.S. Environmental Protection Agency. 