In geographically weighted regression, one must determine a window size which will be used to subset the data locally. Typically, a cross-validation procedure is used to determine a globally optimal window size. Preliminary investigations indicate that the global cross-validation score is heavily influenced by a small number of observations in the dataset. At present, the ramifications of this behaviour in cross-validation are unknown. The research reported here explores the extent to which individual and groups of observations impact optimal window size determination, and whether one can explain why some points are more influential than others. In addition, we strive to examine the impact neighbourhood specification has on model quality in terms of predictive capabilities and the ability of the method to retrieve spatially varying processes. The analysis is based on several datasets and using simulated data in order to compare and validate results. The results provide some practical guidelines for the use of cross-validation.