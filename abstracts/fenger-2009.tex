Air pollution in the industrialised world has in the last 50 years undergone drastic changes. Until after World War II the most important urban compound was sulphur dioxide combined with soot from the use of fossil fuels in heat and power production. When that problem was partly solved by cleaner fuels, higher stacks and flue gas cleaning in urban areas, the growing traffic gave rise to nitrogen oxides and volatile organic compounds and in some areas photochemical air pollution, which may be abated by catalytic converters. Lately the interest has centred on small particles and more exotic organic compounds that can be detected with new sophisticated analytical techniques. Simultaneously with the development in compounds, the time and geographical scale of interest have increased. First to transboundary air pollution, which in decades and on continents can degrade ecosystems, later to the depletion of the ozone layer and especially to the increasing greenhouse effect with climate change that will change the conditions for nature and mankind on the entire globe. The possibilities to study these large scale phenomena have been greatly enhanced by the development of electronic computers that can handle large data sets and calculate various scenarios. All these processes take place in the thin layer of gases around the Earth, the atmosphere. Although the abatement is often restricted to a single aspect, they are often connected and should when possible be treated as whole.