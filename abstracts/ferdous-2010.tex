This paper proposes a multiple discrete continuous nested extreme value (MDCNEV) model to analyze household expenditures for transportation-related items in relation to a host of other consumption categories. The model system presented in this paper is capable of providing a comprehensive assessment of how household consumption patterns (including savings) would be impacted by increases in fuel prices or any other household expense. The MDCNEV model presented in this paper is estimated on disaggregate consumption data from the 2002 Consumer Expenditure Survey data of the United States. Model estimation results show that a host of household and personal socio-economic, demographic, and location variables affect the proportion of monetary resources that households allocate to various consumption categories. Sensitivity analysis conducted using the model demonstrates the applicability of the model for quantifying consumption adjustment patterns in response to rising fuel prices. It is found that households adjust their food consumption, vehicular purchases, and savings rates in the short run. In the long term, adjustments are also made to housing choices (expenses), calling for the need to ensure that fuel price effects are adequately reflected in integrated microsimulation models of land use and travel.