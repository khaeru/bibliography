 AbstractThe number of cities offering bikeshare has increased rapidly, from just a handful in the late 1990s to over 800 currently. This paper provides a review of recent bikeshare literature. Several themes have begun to emerge from studies examining bikeshare. Convenience is the major motivator for bikeshare use. Financial savings has been found to motivate those on a low income and the distance one lives from a docking station is an important predictor for bikeshare membership. In a range of countries, it has been found that just under 50\% of bikeshare members use the system less than once a month. Men use bikeshare more than women, but the imbalance is not as dramatic as private bike riding (at least in low cycling countries). Commuting is the most common trip purpose for annual members. Users are less likely than private cyclists to wear helmets, but in countries with mandatory helmet legislation, usage levels have suffered. Bikeshare users appear less likely to be injured than private bike riders. Future directions include integration with e-bikes, GPS (global positioning system), dockless systems and improved public transport integration. Greater research is required to quantify the impacts of bikeshare, in terms of mode choice, emissions, congestion and health. 