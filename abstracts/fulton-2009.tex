The IEA published “Energy Technology Perspectives” (ETP) in June 2008. That document reports on IEA scenarios for baseline and low-CO$_2$ alternative scenarios to 2050, across the energy economy. The study included creating scenarios for transport, using the IEA Mobility Model (MoMo). This paper reports on the transport-related ETP scenarios and describes the model used in the analysis. According to the ETP Baseline scenario, world transport energy use and CO$_2$ emissions will more than double by 2050. In the most challenging scenario, called “BLUE”, transport emissions are reduced by 70\% in 2050 compared to their baseline level in that year (and about 25\% below their 2005 levels). There are several versions of the BLUE scenario, but all involve: a 50\% or greater improvement in LDV efficiency, 30–50\% improvement in efficiency of other modes (e.g. trucks, ships and aircraft), 25\% substitution of liquid fossil fuels by biofuels, and considerable penetration of electric and/or fuel-cell vehicles. In the second half of this paper, an overview of the MoMo model is provided. Details on the complete analysis are contained in the ETP 2008 document, available at www.iea.org. Details of the LDV fuel economy analysis are contained in a separate paper in this collection.
