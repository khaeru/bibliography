Mobility and accessibility are declining rapidly in most of the developing world. The issues that affect levels of mobility and possibilities for its improvement are varied. They include the rapid pace of motorization, conditions of local demand that far exceed the capacity of facilities, the incompatibility of urban structure with increased motorization, a stronger transport–land use relationship than in developed cities, lack of adequate road maintenance and limited agreement among responsible officials as to appropriate forms of approach to the problem. The rapid rise of motorization presents the question: At what level will it begin to attenuate for given economic and regulatory conditions? Analysts have taken various approaches to this problem, but so far the results are not encouraging. Developing cities have shown significant leadership in vehicle use restrictions, new technologies, privatization, transit management, transit service innovation, transportation pricing and other actions. Only a few, however, have made important strides toward solving the problem. Developing cities have lessons to learn from developed cities as regards roles of new technologies, forms of institutional management and the long term consequences of different de facto policies toward the automobile. These experiences, however, especially in the last category, need to be interpreted very carefully in order to provide useful guidance to cities with, for he most part, entirely different historical experiences in transportation. Continued progress in meeting the needs of the mobility problem in developing cities will focus on: (a) highway building, hopefully used as an opportunity to rationalize access, (b) public transport management improvements, (c) pricing improvements, (d) traffic management, and (e) possibly an emphasis on rail rapid transit based on new revenue techniques.
