Passengers have faced a history of discrimination in transportation systems. Peer transportation companies such as Uber and Lyft present the opportunity to rectify long-standing discrimination or worsen it. We sent passengers in Seattle, WA and Boston, MA to hail nearly 1,500 rides on controlled routes and recorded key performance metrics. Results indicated a pattern of discrimination, which we observed in Seattle through longer waiting times for African American passengers—as much as a 35 percent increase. In Boston, we observed discrimination by Uber drivers via more frequent cancellations against passengers when they used African American-sounding names. Across all trips, the cancellation rate for African American sounding names was more than twice as frequent compared to white sounding names. Male passengers requesting a ride in low-density areas were more than three times as likely to have their trip canceled when they used a African American-sounding name than when they used a white-sounding name. We also find evidence that drivers took female passengers for longer, more expensive, rides in Boston. We observe that removing names from trip booking may alleviate the immediate problem but could introduce other pathways for unequal treatment of passengers.