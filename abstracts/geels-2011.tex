The multi-level perspective (MLP) has emerged as a fruitful middle-range framework for analysing socio-technical transitions to sustainability. The MLP also received constructive criticisms. This paper summarises seven criticisms, formulates responses to them, and translates these into suggestions for future research. The criticisms relate to: (1) lack of agency, (2) operationalization of regimes, (3) bias towards bottom-up change models, (4) epistemology and explanatory style, (5) methodology, (6) socio-technical landscape as residual category, and (7) flat ontologies versus hierarchical levels.