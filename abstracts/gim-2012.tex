To generalize the relationship between density and travel behavior, previous research proceeded with three approaches: metropolitan-level studies describing tendencies on an international scale, area-specific studies extrapolating their outcomes to other areas, and research syntheses pooling descriptive or quantitative outcomes of the studies. However, little research investigated the contextual effect of study areas on the density--travel relationship. Thus, this study conducts meta-analysis to investigate how the magnitude of the relationship differs between two areas that have been frequently studied: the United States and Europe. A pre-test shows that the way of measuring density and travel behavior does not affect the variation in study outcomes, whereas a post-test or sensitivity analysis indicates that the rigor of research designs and statistical techniques affects the variation. The main test finds that the density--travel relationship is significantly stronger in Europe than in the United States. The magnitude difference between the areas is maintained after controlling for confounders, including research design and technical rigor.