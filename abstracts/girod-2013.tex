Travel demand is rising steeply and its contribution to global CO$_2$ emissions is increasing. Different studies have shown possible mitigation through technological options, but so far few studies have evaluated the implications of changing travel behavior on global travel demand, energy use and CO$_2$ emissions. For this study a newly developed detailed passenger transportation model representing technology characteristics as well as key behavioral variables is used. The model allows the reproduction of observed travel demand (1971–2005) in the different world regions and considers income and time rebound effects. Regarding future travel demand, the model allows for an evaluation of the sensitivity for future trends in travel money and time budgets, luxury level, vehicle load and modal split. The study highlights the high relevance of future development in travel behavior for climate policy. A consistent combination of different behavioral changes towards a more climate friendly travel behavior is modeled to reduce CO$_2$ emissions towards the end of this century by around 50\% compared to the baseline.
