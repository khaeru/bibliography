This short essay first reviews the pioneers of energy transition research both in terms of data as well as theories. Three major insights that have emerged from this nascent research fields are summarized highlighting the importance of energy end-use and services, the lengthy process of transitions, as well as the patterns that characterize successful scale up of technologies and industries that drive historical energy transitions. The essay concludes with cautionary notes also derived from historical experience. In order to trigger a next energy transition policies and innovation efforts need to be persistent and continuous, aligned, as well as balanced. It is argued that current policy frameworks in place invariably do not meet these criteria and need to change in order to successfully trigger a next energy transition towards sustainability.