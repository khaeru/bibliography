The release of emails from a server at the University of East Anglia's Climate Research Unit (CRU) in November 2009 and the following climategate controversy have become a topic for interpretation in the social sciences. This article picks out some of the most visible social science comments on the affair for discussion. These comments are compared to an account of what can be seen as problematic practices by climate scientists. There is general agreement in these comments that climate science needs more openness and transparency. But when evaluating climategate a variety of responses is seen, ranging from the apologetic to the highly critical, even condemning the practices in question. It is argued that reluctance to critically examine the climategate affair, including suspect practices of scientists, has to do with the nature of the debate which is highly politicized. A call is made for more reflection on this case which should not be closed off because of political expediency.