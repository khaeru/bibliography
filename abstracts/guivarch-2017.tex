Scenario techniques are a teeming field in energy and environmental research and decision making. This Thematic Issue (TI) highlights quantitative (computational) methods that improve the development and use of scenarios for dealing with the dual challenge of complexity and (deep) uncertainty. The TI gathers 13 articles that describe methodological innovations or extensions and refinements of existing methods, as well as applications that demonstrate the potential of these methodological developments. The TI proposes two methodological foci for dealing with the challenges of (deep) uncertainty and complexity: diversity and vulnerability approaches help tackle uncertainty; multiple-objective and multiple-scale approaches help address complexity; whereas some combinations of those foci can also be applied. This overview article to the TI presents the contributions gathered in the TI, and shows how they individually and collectively bring new capacity to scenarios techniques to deal with complexity and (deep) uncertainty.