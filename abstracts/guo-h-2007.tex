International Vehicle Emissions (IVE) model funded by U.S. Environmental Protection Agency (USEPA) is designed to estimate emissions from motor vehicles in developing countries. In this study, the IVE model was evaluated by utilizing a dataset available from the remote sensing measurements on a large number of vehicles at five di erent sites in Hangzhou, China, in 2004 and 2005. Average fuel-based emission factors derived from the remote sensing measurements were compared with corresponding emission factors derived from IVE calculations for urban, hot stabilized condition. The results show a good agreement between the two methods for gasoline passenger cars’ HC emission for all IVE subsectors and technology classes. In the case of CO emissions, the modeled results were reasonably good, although systematically underestimate the emissions by almost 12\%–50\% for di erent technology classes. However, the model totally overestimated NO$_X$ emissions. The IVE NO$_X$ emission factors were 1.5–3.5 times of the remote sensing measured ones. The IVE model was also evaluated for light duty gasoline truck, heavy duty gasoline vehicles and motor cycles. A notable result was observed that the decrease in emissions from technology class State II to State I were overestimated by the IVE model compared to remote sensing measurements for all the three pollutants. Finally, in order to improve emission estimation, the adjusted base emission factors from local studies are strongly recommended to be used in the IVE model.
