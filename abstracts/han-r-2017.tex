In response to the growing need to reduce carbon emissions, it is necessary to explore the design of carbon trading mechanisms and discuss allocation options for the transport sector. This paper examines the allocation of carbon quotas with the introduction of an emissions trading scheme (ETS) in the Chinese road transport sector. Aiming to simulate the allocation of carbon emissions quotas, we forecast vehicle possession using a gray forecast model and trend extrapolation; consider the carbon dioxide (CO2) emissions of the transport sector using a top-down approach; and design three policy scenarios. We provide the following findings. First, vehicle possession in the road transport sector and carbon emissions both display an increasing trend, reaching 180 million units and 6.6 billion tons by 2020, respectively. Second, the road transport sector has the largest carbon quota under the benchmark scenario and the smallest under a low-carbon scenario. The difference between these two scenarios is 2.7 billion tons of carbon emissions. Finally, we design a carbon emissions trading mechanism for the transport sector based on China's special development period, and provide a sensitivity analysis.