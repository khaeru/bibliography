Facing heavy air pollution, China needs to transition to a clean and sustainable energy system, especially in the power and transport sectors, which contribute the highest greenhouse gas (GHG) emissions. The core of an energy transition is energy substitution and energy technology improvement. In this paper, we forecast the levelized cost of electricity (LCOE) for power generation in 2030 in China. Cost-emission effectiveness of the substitution between new energy vehicles and conventional vehicles is also calculated in this study. The results indicate that solar photovoltaic (PV) and wind power will be cost comparative in the future. New energy vehicles are more expensive than conventional vehicles due to their higher manufacturer suggested retail price (MSRP). The cost-emission effectiveness of the substitution between new energy vehicles and conventional vehicles would be $96.7/ton or $114.8/ton. Gasoline prices, taxes, and vehicle insurance will be good directions for policy implementation after the ending of subsidies.