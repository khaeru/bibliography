The primary objective of this paper is to make a connection between the Singapore story of land transport policy development and the pathway towards sustainable transport planning. Its innovation is to allow a parallel growth in motorization and public transit. The Singapore experience shows how a range of well-coordinated policies including efforts to control the number of cars in both ownership and usage, and at the same time increase the availability and rider-ship of public transit contribute to a sustainable transport system. The Singapore model provides a valuable reference for not only developing an alternative approach towards sustainable transport in countries where motorization is desired but also for understanding the various parameters important to the formulation of management policies. In particular, the paper contends the Singapore experience provides a model for Asian nations and cities.
