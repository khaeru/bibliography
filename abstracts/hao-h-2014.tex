With China’s urbanization and motorization, greenhouse gas (GHG) emissions from urban passenger transport increased rapidly over recent years. As we estimated, China’s urban passenger transport associated motorized travel, energy consumption and lifecycle \{GHG\} emissions reached 2815 billion passenger kilometers (pkm), 77 million tons of oil equivalent (toe) and 335 million ton \{CO2\} equivalent in 2010, over half of which were located in eastern provinces. Over national level, \{GHG\} emissions by private passenger vehicles, business passenger vehicles, taxis, motorcycles, E-bikes, transit buses and urban rails accounted for 57.7%, 13.0%, 7.7%, 8.6%, 1.8%, 10.5% and 0.7% of the total. Significant regional disparity was observed. The province-level per capita \{GHG\} emissions ranged from 285 kg/capita in Guizhou to 1273 kg/capita in Beijing, with national average of 486 kg/capita. Depending on province context and local policy orientation, the motorization pathways of China’s several highest motorized provinces are quite diverse. We concluded that motorization rate and transport structure were the substantial factors determining urban passenger transport associated \{GHG\} emissions. Considering the great potential of urban passenger transport growth in China, policies guiding the optimization of transport structure should be in place with priority in eastern provinces.