Energy efficiency is one of the key factors affecting energy consumption and greenhouse gas (GHG) emissions. By focusing on China’s transport sector, this study comprehensively reviews and compares the energy efficiency performance of passenger vehicles, light-duty commercial vehicles, commercial road transport, commercial water transport, aviation transport and railway transport, and identifies the opportunities for further energy efficiency improvements. It is found that railway transport exhibited the greatest improvement in energy efficiency during the past decade, which was mainly driven by progress in its electrification. Passenger vehicles have also experienced considerable energy efficiency improvements, which can be mainly attributed to the establishment of mandatory fuel consumption standards. In contrast, commercial road transport has shown the least improvement, due to insufficient policy implementations. Based on the analysis, it is recommended that, as China’s present policy framework to improve energy efficiency in the transport sector is generally effective, it should be consistently maintained and successively improved. Electrification represents a major opportunity for improvement of energy efficiency in the transport sector. Such potential should be fully tapped for all transport modes. Greater effort should be put into improving the energy efficiency of commercial road transport. The policy instruments utilized to improve the energy efficiency of heavy-duty vehicles should be as intensive and effective as the policy instruments for passenger vehicles.