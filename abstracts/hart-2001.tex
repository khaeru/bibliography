The history of antitrust policy in the \{US\} as it relates to technological innovation exhibits major swings every few decades between favoring concentration and favoring deconcentration. This paper sketches for each period the contending ideas that frame antitrust-technology policy debates, the salience of these ideas in the larger antitrust policy process, the institutions for agenda-setting and decision-making in this area, the policy decisions themselves, and (more speculatively) the impacts of these decisions on technological innovation and industrial development. The paper concludes with a preliminary attempt to identify the cyclical, secular, and static processes that have shaped the history of this policy area and to use this analysis to inform future policy-makers.
