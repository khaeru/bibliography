Measures of the value of public investments are critical inputs into the policy process, and aggregate production and cost functions have become the dominant methods of evaluating these benefits. This paper examines the limitations of these approaches in light of applied production and spatial equilibrium theories. A spatial equilibrium model of an economy with nontraded, localized public goods like infrastructure is proposed, and a method for identifying the role of public capital in firm production and household preferences is derived. Empirical evidence from a sample of large US cities suggests that public capital provides significant marginal benefits. But the marginal productivity of public capital is low, and aggregate city willingness to pay for large increases in public capital is less than their cost. The paper concludes with a brief discussion of the political economy of urban infrastructure investment.
