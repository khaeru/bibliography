This paper analyses the impacts of the gasoline vehicle replacement programme with EVs at different penetration rates on petroleum and electricity sectors and their CO2 emissions. The study utilises a top-down-type Environmental Input–Output (EI–O) model. Our results show that the replacement of gasoline cars with EVs causes greater impacts on total gasoline production than on total electricity generation. For example, at 5%, 20%, 50%, 70% and 100% gasoline vehicle replacement with EVs, the total gasoline production decreases by 1.66%, 6.65%, 16.62%, 23.27% and 33.24% in policy scenario 1, while the total electricity production only increases by 0.71%, 2.82%, 7.05%, 9.87% and 14.10%. Our study confirms that the gasoline vehicle replacement with EVs, powered by 80% coal, has no effect on overall emissions. The CO2 emissions reduction in the petroleum sector is offset by the increase in CO2 emissions in the electricity sector, leaving the national CO2 emissions unchanged. By decarbonising the electricity sector, i.e. using 30% less coal in electricity generation mix, the total CO2 emissions will be reduced by 28% (from 10,953 to 7870Mt CO2) on the national level. The gasoline vehicle replacement programme with EVs, powered by 50% coal-based electricity, helps reduce CO2 emissions in petroleum sector and contributes zero or a very small proportion of additional CO2 emissions to the electricity sector (policy scenario 2 and 3). We argue that EVs can contribute to a reduction of petroleum dependence, air quality improvement and CO2 emission reduction only when their introduction is accompanied by aggressive electricity sector decarbonisation.