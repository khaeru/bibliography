China researchers make frequent use of official Chinese statistics, but these statistics are often not well understood. This article clarifies three major issues that affect a wide range of Chinese statistics: changes over time to the sectoral classification system, to the ownership classification system, and to the coverage of the industry sector. Many of these changes have gone unnoticed or are only poorly recognized, leaving the researcher puzzled about varying labels, apparently inconsistent data, and discontinued time series. A second part of the article provides an overview of the wealth of Chinese statistics now available. It points out the limitations of some of these sources and provides references to a secondary literature that discusses the meaning and quality of particular Chinese statistics.
