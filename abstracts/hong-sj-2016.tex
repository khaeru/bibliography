On August 15, 2008, the 60th anniversary of the nation׳s founding, South Korea declared “Low Carbon, Green Growth” as its new national agenda, and announced the country׳s commitment to reduce its greenhouse gas(GHG) emissions by 30% by 2020, relative to business-as-usual (BAU) scenario. According to the Korean GHG reduction roadmap in 2014, the transportation sector was allocated the highest GHG reduction rate of 34.3% in order to meet the national target. This study examined the effectiveness of policies that the South Korean government has imposed on the transportation sector and analyzed the ripple effect in terms of energy and environmental aspects if the policies are maintained until 2050 using Long-range Energy Alternatives Planning system (LEAP) model. The study considered five policies, in terms of scenarios: improved fuel efficiency (IFE), green cars distribution (GC, CGC), public transportation shift (PTS), and modal shift reinforcement (MS). To distinguish the effects of various eco-friendly vehicle models, the green car distribution scenario was subdivided into two: green car scenario and competitive green car scenario. The green car (GC) scenario is focused on clean diesel and hybrid cars, and the competitive green car (CGC) scenario centered on hydrogen fuel cell and electric vehicles. According to the scenario analysis conducted in the current study, the final energy and GHG reduction policies that the South Korean government has imposed on the transportation sector will reduce the final energy demand by 25.2–25.5%, and reduce GHG emissions by 21.3–21.6% in 2020, relative to BAU scenario. However, this number does not meet the national GHG reduction target rate of 34.3% in the Korean transportation sector. Therefore, more powerful and effective policies are needed to achieve the national target vis-à-vis GHG emissions reduction in the transportation sector.