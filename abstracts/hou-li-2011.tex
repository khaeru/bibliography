This paper analyses the accessibility implications of the development of expressways and inter-city railways in the Greater Pearl River Delta (GPRD) over the period 1990–2020. Average travel time was firstly reduced by expressway development; and it will be reduced further by the introduction of the inter-city rail system in 2011. The unevenness in regional accessibility remained relatively high during the initial stage of expressway development, but later expansion brought more balanced accessibility landscapes. The first stage (2010–2020) of inter-city railway development will raise the accessibility inequality. Its later effects, however, remain to be seen. Convenience in transport connections is associated with the spatial pattern of industrialization. In addition, accessibility improvement is tied to the direction of city-region development, as exemplified by Guangzhou’s choice of Nansha, the city’s outer port, as development focus.