Life cycle assessment (LCA) is a method to evaluate the environmental impacts of technologies from cradle to grave. However, LCAs are commonly defined in terms of the consumption of a single unit of a product and thus ignore scaling issues in large-scale deployment of technologies. Such product-level LCAs often do not consider capital manufacturing capacity and supply chain bottlenecks that may hinder the rapid, widespread uptake of emerging technologies entering the market; emerging technologies often require the expansion of existing supply chains or the development of entirely new supply chains, such as the manufacturing of novel materials. As a result, such LCA studies are limited in their ability to realistically assess impacts at the macro-scale and thus to guide large-scale decisions. In this work, we present ECOPT2, a generalized adaptable model that combines these constraints to the LCA approach using a mathematical programming approach and dynamic stock modeling. ECOPT2 combines LCA factors with transition scenarios from energy systems models to determine the environmentally optimal deployment of new technologies while accounting for material circularity constraints and barriers to uptake. We also introduce the structure of the software tool and demonstrate its features using a stylized vehicle electrification scenario.
