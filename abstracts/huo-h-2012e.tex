This paper, which focuses on diesel trucks, is the third in a series of three papers published in Atmospheric Environment to understand vehicle emissions in China by conducting on-board emission measurements. Diesel trucks are a significant source of emissions in ambient air, especially for NO$_X$. Recently, China announces an aggressive target to reduce national NO$_X$ emissions by 10\% from 2010 to 2015 in the “Twelfth Five-Year Plan (2011–2015)” and diesel vehicles are identified as a key target for NO$_X$ control. However, the understanding of the real-world emissions of diesel trucks is limited. In this study, we measured HC, CO, NO$_X$, and PM$_{2.5}$ emissions from 175 diesel trucks of different sizes and technologies in five Chinese cities during 2007 and 2011, and generated emission factors on the basis of the measurements. The results show that the HC, CO, and PM$_{2.5}$ emission factors have been reduced significantly as the emission standards become more stringent from Euro 0 to Euro IV, but the NO$_X$ emission factors change differently. Euro II trucks have 3–6\% higher NO$_X$ emission levels than Euro I technologies and Euro III trucks fail to show a reduction as regulated by the standards. More stringent NO$_X$ requirements (e.g. Euro IV) for diesel vehicles need to be enforced. The comparison with the emission factors used in recent emission inventory studies shows that these inventories may have overestimated or underestimated diesel emissions for the years after 2006. This study emphasizes the importance of conducting local measurement research to improve the accuracy of the estimates of mobile emissions in China.
