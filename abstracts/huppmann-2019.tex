The MESSAGE Integrated Assessment Model (IAM) developed by IIASA has been a central tool of energy-environment-economy systems analysis in the global scientific and policy arena. It played a major role in the Assessment Reports of the Intergovernmental Panel on Climate Change (IPCC); it provided marker scenarios of the Representative Concentration Pathways (RCPs) and the Shared Socio-Economic Pathways (SSPs); and it underpinned the analysis of the Global Energy Assessment (GEA). Alas, to provide relevant analysis for current and future challenges, numerical models of human and earth systems need to support higher spatial and temporal resolution, facilitate integration of data sources and methodologies across disciplines, and become open and transparent regarding the underlying data, methods, and the scientific workflow. In this manuscript, we present the building blocks of a new framework for an integrated assessment modeling platform; the “ecosystem” comprises: i) an open-source GAMS implementation of the MESSAGE energy++ system model integrated with the MACRO economic model; ii) a Java/database back-end for version-controlled data management, iii) interfaces for the scientific programming languages Python & R for efficient input data and results processing workflows; and iv) a web-browser-based user interface for model/scenario management and intuitive “drag-and-drop” visualization of results. The framework aims to facilitate the highest level of openness for scientific analysis, bridging the need for transparency with efficient data processing and powerful numerical solvers. The platform is geared towards easy integration of data sources and models across disciplines, spatial scales and temporal disaggregation levels. All tools apply best-practice in collaborative software development, and comprehensive documentation of all building blocks and scripts is generated directly from the GAMS equations and the Java/Python/R source code.