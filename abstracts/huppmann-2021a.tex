The open-source Python package pyam provides a suite of features and methods for the analysis, validation and visualization of reference data and scenario results generated by integrated assessment models, macro-energy tools and other frameworks in the domain of energy transition, climate change mitigation and sustainable development.
It bridges the gap between scenario processing and visualisation solutions that are "hard-wired" to specific modelling frameworks and generic data analysis or plotting packages.

The package aims to facilitate reproducibility and reliability of scenario processing, validation and analysis by providing well-tested and documented methods for working with timeseries data in the context of climate policy and energy systems.
It supports various data formats, including sub-annual resolution using continuous time representation and "representative timeslices".

The pyam package can be useful for modelers generating scenario results using their own tools as well as researchers and analysts working with existing scenario ensembles such as those supporting the IPCC reports or produced in research projects.
It is structured in a way that it can be applied irrespective of a user's domain expertise or level of Python knowledge, supporting experts as well as novice users.

The code base is implemented following best practices of collaborative scientific-software development.
This manuscript describes the design principles of the package and the types of data which can be handled.
The usefulness of pyam is illustrated by highlighting several recent applications.
