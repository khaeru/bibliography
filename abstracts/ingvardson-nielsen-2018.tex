Cities worldwide are implementing modern transit systems to improve mobility in the increasingly congested metropolitan areas. Despite much research on the effects of such systems, a comparison of effects across transit modes and countries has not been studied comprehensively. This paper fills this gap in the literature by reviewing and comparing the effects obtained by 86 transit systems around the world, including Bus Rapid Transit (BRT), Light Rail Transit (LRT), metro and heavy rail transit systems. The analysis is twofold by analysing (i) the direct operational effects related to travel time, ridership and modal shifts, and (ii) the indirect strategic effects in terms of effects on property values and urban development. The review confirms the existing literature suggesting that BRT can attract many passengers if travel time reductions are significantly high. This leads to attractive areas surrounding the transit line with increasing property values. Such effects are traditionally associated with attractive rail-based public transport systems. However, a statistical comparison of 41 systems did not show significant deviations between effects on property values resulting from BRT, LRT and metro systems, respectively. Hence, this paper indicates that large strategic effects can be obtained by implementing BRT systems at a much lower cost.