Today the halls of Technology, Entertainment, and Design (TED) and Davos reverberate with optimism that hacking, brainstorming, and crowdsourcing can transform citizenship, development, and education alike. This article examines these claims ethnographically and historically with an eye toward the kinds of social orders such practices produce. This article focuses on a hackathon, one emblematic site of social practice where techniques from information technology (IT) production become ways of remaking culture. Hackathons sometimes produce technologies, and they always, however, produce subjects. This article argues that the hackathon rehearses an entrepreneurial citizenship celebrated in transnational cultures that orient toward Silicon Valley for models of social change. Such optimistic, high-velocity practice aligns, in India, with middle-class politics that favor quick and forceful action with socially similar collaborators over the contestations of mass democracy or the slow construction of coalition across difference.