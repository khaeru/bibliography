Longitudinal cohort analysis, where the cohorts are constituted by different generations (birth-years), proved necessary for explaining the post-war development of car ownership in Sweden. This approach was applied in the forecasting model developed at the Swedish Road and Traffic Research Institute for the Swedish National Road Administration. The model assumes the "propensity to enter" into car ownership (with its counterpart "exit propensity") to be the primary dependent variable. The car is regarded as an individual good, and separate forecasts are made for male and female car ownership. In the main scenario adopted, it is forecast that the total number of passenger cars will increase from 3.32 million in 1984 to 3.84 million by the year 2000 and to 4.10 million by the year 2010, and 90 per cent of the increase is attributed to women.
