 This article explores rural household food consumption behaviour in China using a large household data set from Jilin Province. Data are classified into four main food groups—grain, vegetable products, animal products and other foods. A household food demand system, incorporating four household characteristics, is estimated using an LA-AIDS model, assuming a three-stage budgeting procedure. Expenditure elasticities for a range of food groups are estimated, with a particular focus on animal products. The inclusion of household characteristics did not have a big impact on the elasticity values in any of the three stages of the budgeting process. The total expenditure elasticity for grain (Stage II) was 0.64, suggesting substantial future growth in household demand for fine grains such as rice and wheat, as per capita incomes continue to grow in rural areas. The highest conditional and total expenditure elasticity values were for the animal products (Stage II) group, 1.22 and 0.76, respectively. Within this group the elasticities were highest for the meat sub-group at 1.14 and 0.87, respectively, suggesting an almost proportionate increase in demand as household incomes grow. Added demand pressures from animal production will likely keep grain policy high on the political agenda. 