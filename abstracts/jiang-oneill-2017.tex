The new scenario process for climate change research includes the creation of Shared Socioeconomic Pathways (SSPs) describing alternative societal development trends over the coming decades. Urbanization is a key aspect of development that is relevant to studies of mitigation, adaptation, and impacts. Incorporating urbanization into the SSPs requires a consistent set of global urbanization projections that cover long time horizons and span a full range of uncertainty. Existing urbanization projections do not meet these needs, in particular providing only a single scenario over the next few decades, a period during which urbanization is likely to be highly dynamic in many countries. We present here a new, long-term, global set of urbanization projections at country level that cover a plausible range of uncertainty. We create SSP-specific projections by choosing urbanization outcomes consistent with each SSP narrative. Results show that the world continues to urbanize in each of the SSPs but outcomes differ widely across them, with urbanization reaching 60%, 79%, and 92% by the end of century in SSP3, SSP2, and SSP1/SSP4/SSP5, respectively. The degree of convergence in urbanization across countries also differs substantially, with largely convergent outcomes by the end of the century in SSP1 and SSP5 and persistent diversity in SSP3. This set of global, country-specific projections produces urbanization pathways that are typical of regions in different stages of urbanization and development levels, and can be extended to further elaborate assumptions about the styles of urban growth and spatial distributions of urban people and land cover occurring in each SSP.