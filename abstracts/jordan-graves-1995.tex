Increasing manufacturing flexibility is a key strategy for efficiently improving market responsiveness in the face of uncertain future product demand. Process flexibility results from being able to build different types of products in the same plant or production facility at the same time. In Part I of this paper, we develop several principles on the benefits of process flexibility. These principles are that 1) limited flexibility (i.e., each plant builds only a few products), configured in the right way, yields most of the benefits of total flexibility (i.e., each plant builds all products) and 2) limited flexibility has the greatest benefits when configured to chain products and plants together to the greatest extent possible. In Part II, we provide analytic support and justification for these principles. Based on a planning model for assigning production to plants, we demonstrate that, for realistic assumptions on demand uncertainty, limited flexibility configurations (i.e., how products are assigned to plants) have sales benefits that are approximately equivalent to those for total flexibility. Furthermore, from this analysis we develop a simple measure for the flexibility in a given product-plant configuration. Such a measure is desirable because of the complexity of computing expected sales for a given configuration. The measure is ∏(M*), the maximal probability over all groupings or sets of products (M) that there will be unfilled demand for a set of products while simultaneously there is excess capacity at plants building other products. This measure is easily computed and can be used to guide the search for good limited flexibility configurations.
