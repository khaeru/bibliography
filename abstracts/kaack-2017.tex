While many forecasters are moving toward generating probabilistic predictions, energy forecasts typically still consist of point projections and scenarios without associated probabilities. Empirical density forecasting methods provide a probabilistic amendment to existing point forecasts. Here we lay the groundwork for evaluating the performance of these methods in the data-scarce setting of long-term forecasts. Results can give policy analysts and other users confidence in estimating forecast uncertainties with empirical methods. Hundreds of organizations and analysts use energy projections, such as those contained in the US Energy Information Administration (EIA)'s Annual Energy Outlook (AEO), for investment and policy decisions. Retrospective analyses of past AEO projections have shown that observed values can differ from the projection by several hundred percent, and thus a thorough treatment of uncertainty is essential. We evaluate the out-of-sample forecasting performance of several empirical density forecasting methods, using the continuous ranked probability score (CRPS). The analysis confirms that a Gaussian density, estimated on past forecasting errors, gives comparatively accurate uncertainty estimates over a variety of energy quantities in the AEO, in particular outperforming scenario projections provided in the AEO. We report probabilistic uncertainties for 18 core quantities of the AEO 2016 projections. Our work frames how to produce, evaluate, and rank probabilistic forecasts in this setting. We propose a log transformation of forecast errors for price projections and a modified nonparametric empirical density forecasting method. Our findings give guidance on how to evaluate and communicate uncertainty in future energy outlooks.
