The transition to a cleaner and more cost-efficient electricity system in China is political-economic as well as technological. An example is the reform of China's method of dispatching power plants, which potentially affects the economic relationships between consumers and producers, between grid and generating companies, and between central and provincial governments. Historically, coal-fired power plants in China all received roughly the same number of operating hours, regardless of efficiency or cost. In 2007, Chinese government agencies began to pilot “energy efficient dispatch,” which requires that generators be dispatched on the basis of thermal efficiency. Using a case study of Guangxi Zhuang Autonomous Region in southern China, we evaluated potential energy and cost savings from a change to energy efficient dispatch. We found that the savings are at best relatively small, because large, efficient generators already account for a significant share of total generation. Moreover, as an administrative policy that does not change economic incentives, energy efficient dispatch exacerbates imbalances and center-provincial tensions in the current system. We argue that incentive-based dispatch reform is likely to produce better outcomes, and that the keys to this reform are empowering an independent regulator with pricing authority and establishing a formal, transparent ratemaking process.
