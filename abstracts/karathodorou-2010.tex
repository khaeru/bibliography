Much of the empirical literature on fuel demand presents estimates derived from national data which do not permit any explicit consideration of the spatial structure of the economy. Intuitively we would expect the degree of spatial concentration of activities to have a strong link with transport fuel consumption. The present paper addresses this theme by estimating a fuel demand model for urban areas to provide a direct estimate of the elasticity of demand with respect to urban density. Fuel demand per capita is decomposed into car stock per capita, fuel consumption per kilometre and annual distance driven per car per year. Urban density is found to affect fuel consumption, mostly through variations in the car stock and in the distances travelled, rather than through fuel consumption per kilometre.