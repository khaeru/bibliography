Motorization is a self-reinforcing process, which involves positive feedback. Using a simple model of bi-modal transportation system, it is shown that a transportation system may ually be dominated by the automobile, or may equilibrate at a grossly inefficient point due to social dilemmas associated with automobile use. Furthermore, in areas with well-developed public transit, road capacity addition can be detrimental not only to the public transit but also to the automotive transportation. Travel demand management (TDM) measures are conceived in this study as mechanisms to pr the divergence of the motorization process away from a social optimum due to its positive feedback and social dilemmas. A simple model of bi-modal transportation system and a cellular automata model of individuals’ TDM compliance behavior are combined in the study to see how TDM measures that rely on individuals’ voluntary cooperation may be effective. Results of simulation analyses based on the model are presented.
