China's national leaders have recently made a priority of changing lanes from a pollution-intensive, growth-at-any-cost model to a resource-efficient and sustainable one. The immense challenges of rapid urbanization are one aspect of the problem. Central-local government relations are another source of challenges, since the central government's green agenda does not always find willing followers at lower levels. This paper identifies barriers to a more comprehensive implementation of environmental policies at the local level in China's urban areas and suggests ways to reduce or remove them. The research focuses particularly on the reasons for the gap between national plans and policy outcomes. Although environmental goals and policies at the national level are quite ambitious and comprehensive, insufficient and inconsistent local level implementation can hold back significant improvements in urban environmental quality. By analyzing local institutional and behavioral obstacles and by highlighting best-practice examples from China and elsewhere, the paper outlines options that can be used at the national and local levels to close the local “environmental implementation gap.” The findings emphasize the need to create additional incentives and increase local implementation capacities.
