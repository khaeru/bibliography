This article presents the synthesis of results from the Stanford Energy Modeling Forum Study 27, an inter-comparison of 18 energy-economy and integrated assessment models. The study investigated the importance of individual mitigation options such as energy intensity improvements, carbon capture and storage (CCS), nuclear power, solar and wind power and bioenergy for climate mitigation. Limiting the atmospheric greenhouse gas concentration to 450 or 550 ppm CO2 equivalent by 2100 would require a decarbonization of the global energy system in the 21st century. Robust characteristics of the energy transformation are increased energy intensity improvements and the electrification of energy end use coupled with a fast decarbonization of the electricity sector. Non-electric energy end use is hardest to decarbonize, particularly in the transport sector. Technology is a key element of climate mitigation. Versatile technologies such as CCS and bioenergy are found to be most important, due in part to their combined ability to produce negative emissions. The importance of individual low-carbon electricity technologies is more limited due to the many alternatives in the sector. The scale of the energy transformation is larger for the 450 ppm than for the 550 ppm CO2e target. As a result, the achievability and the costs of the 450 ppm target are more sensitive to variations in technology availability.