Integrated assessments of how climate policy interacts with energy-economy systems can be performed by a variety of models with different functional structures. In order to provide insights into why results differ between models, this article proposes a diagnostic scheme that can be applied to a wide range of models. Diagnostics can uncover patterns of model behavior and indicate how results differ between model types. Such insights are informative since model behavior can have a significant impact on projections of climate change mitigation costs and other policy-relevant information. The authors propose diagnostic indicators to characterize model responses to carbon price signals and test these in a diagnostic study of 11 global models. Indicators describe the magnitude of emission abatement and the associated costs relative to a harmonized baseline, the relative changes in carbon intensity and energy intensity, and the extent of transformation in the energy system. This study shows a correlation among indicators suggesting that models can be classified into groups based on common patterns of behavior in response to carbon pricing. Such a classification can help to explain variations among policy-relevant model results.