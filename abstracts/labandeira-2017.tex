Price elasticities of energy demand have become increasingly relevant in estimating the socio-economic and environmental effects of energy policies or other events that influence the price of energy goods. Since the 1970s, a large number of academic papers have provided both short and long-term price elasticity estimates for different countries using several models, data and estimation techniques. Yet the literature offers a rather wide range of estimates for the price elasticities of demand for energy. This paper quantitatively summarizes the recent, but sizeable, empirical evidence to facilitate a sounder economic assessment of (in some cases policy-related) energy price changes. It uses meta-analysis to identify the main factors affecting short and long term elasticity results for energy, in general, as well as for specific products, i.e., electricity, natural gas, gasoline, diesel and heating oil.