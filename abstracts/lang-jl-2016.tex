Multi-year (1999–2014) vehicular unregulated pollutants emissions in China, including SO2, CH4, N2O, NH3, Indeno(1,2,3-cd)pyrene (IPY), Benzo(k)fluoranthene (BkF), Benzo(b)fluoranthene (BbF), Benzo(a)pyrene (BaP), dioxins and furans, were estimated based on emission factors calculated by COPERT. The inter-annual trends, correlation with \{GDP\} and population, spatial distribution characteristics, contributions from various vehicle types for the ten pollutants emissions were analyzed. Results showed that the emissions of the above ten pollutants changed from approximately 576.9 Gg, 130.0 Gg, 8.1 Gg, 2.1 Gg, 1.0 Mg, 1.1 Mg, 1.4 Mg, 0.5 Mg, 7.4 g and 15.6 g in 1999 to 193.8 Gg, 171.1 Gg, 79.1 Gg, 117.8 Gg, 3.5 Mg, 6.7 Mg, 6.8 Mg, 2.9 Mg, 37.6 g and 79.1 g in 2014, respectively. Implementation of stringent sulfur content limit during the past decade reduced approximately 94.4\% of the \{SO2\} emission in 2014. \{CH4\} and \{N2O\} increased from 1999 to 2011, but began to decrease since 2012; \{NH3\} emission had the highest annual average change rate (35.5\%) from 1999 to 2014; PAHs, dioxins and furans increased continuously during the past decade. The vehicular emissions were higher in Guangdong, Shandong, Henan, Jiangsu, Zhejiang and Hebei. Good linear correlation between vehicular emissions and \{GDP\} was found (except SO2); in the provinces/municipalities with higher population density, the emission density was also larger, indicating more significant vehicular emissions' potential damage on human health. \{HDT\} and PC, \{PC\} and MC, \{PC\} and \{BUS\} were the major contributors to SO2, CH4, \{N2O\} emissions, respectively. In 2014, \{PC\} was the dominant contributor to \{NH3\} emission; PC, \{BUS\} and \{HDT\} had higher fraction in the total \{IPY\} and BaP emissions; \{HDT\} was the major contributor to BkF and BbF emissions. In addition, the uncertainties of estimated emissions were also analyzed based on Monte Carlo simulation.