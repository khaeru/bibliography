Cross-country statistics have revealed steady growth in the number of motorcycles in many less advanced economic countries (LAEC) with emerging economies due to increased urbanisation and personal wealth. In contrast, an opposite trend is occurring in advanced economic countries (AEC), with cars replacing motorcycles as income grows. Motor vehicle crashes and injuries are an inevitable consequence of a high motorcycle population. This study focused on understanding how economic growth affects the motorcycle to passenger car (MPC) ownership ratio and what factors underlie this relationship. The data used in this analysis contained a sample of 80 countries at various levels of economic developmental growth over the 48-year period between 1963 and 2010. The results pointed to an inverted U-shaped relationship between the MPC ownership ratio and the per capita Gross Domestic Product (GDP). Generally, the MPC ownership ratio increased with income at a lower level and decreased with income at a higher level. The evidence indicated that urbanisation, the total road length per thousand population, and a proxy for purchasing power with regard to vehicle purchases were the underlying factors that contributed to this relationship.
