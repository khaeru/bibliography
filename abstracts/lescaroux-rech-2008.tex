Prompted by the recent surge in light oil product consumption, this paper analyses the demand for non-commercial motor fuel and proposes a longrun forecasting model. In doing so, our aim is to be able to reproduce a few key stylized facts observed in secular evolutions of the motor fuel intensity of GDP and related notably to the derived nature of oil demand. Using a database covering 77 countries over the 1986-1998 period, we explain sequentially the stock of private vehicles per capita and fuel consumption per vehicle. The former is expressed as an S-shaped function of real per-capita income, which takes into account the dynamics specific to the dissemination of a durable good in a population. By explicitly considering the distinct phases of the development of the automobile market, our approach enables us to propose an explanation to the space-time variability in long-run income elasticities reported in the literature - especially its decline as per-capita income increases and the resulting gap between elasticities in emerging countries compared to developed countries. Our two-equation model also enables us to reproduce the "bell" shaped curve of the motor fuel intensity of GDP as a function of per-capita income, as well as the other principal properties of resource intensity-of-use linked to the process of dematerialization which, for any country, follows the industrialization period.