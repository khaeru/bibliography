On the basis of local concavity assumptions, Exact Affine Stone Index implicit Marshallian demand system is adopted to analyze Engel Curves and price elasticity for Chinese urban household. Our results show that a demand system rank can be polynomials or splines of any order which support the conclusion of Lewbel and Pendakur (2009). Engel Curves are affected by gender and education of householder, the number of minor children and adults. Budget shares of food, clothing and transportation–communication decrease, and other budget shares increase from 1995 to 2007. There are different influences between all categories of price elasticities, most notably dwelling price. In addition, the quickly rising price and slow growth spending hamper improvement of welfare from 2002 to 2012.
