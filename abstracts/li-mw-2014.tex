We examine the effects of recently announced energy policies in mainland China on air quality in both China and the U.S. in 2030. China is the largest contributor to global anthropogenic emissions of air pollutants, especially the precursors to ozone and fine particulate matter (PM$_{2.5}$) such as nitrogen oxides (NO$_X$) and sulfur dioxide (SO$_2$). Efforts to limit coal use in China under the country’s National Air Pollution Action Plan will reduce these air pollutants. Control efforts are expected to not only decrease the concentration of ozone and PM$_{2.5}$ locally in China, but also reduce the trans-Pacific transport of air pollutants to the U.S. We couple an energy-economic model with sub-national detail for China (the China Regional Energy Model, or C-REM) to a global atmospheric chemistry model (GEOS-Chem) to assess air pollution reductions under an energy policy scenario relative to a no policy baseline scenario. Future Chinese anthropogenic emissions are predicted by C-REM under a national energy policy scenario which achieves a 20\% reduction in energy intensity from 2012 to 2017 by targeting fossil fuel use nationwide as specified in the National Air Pollution Action Plan and also meets the Plan’s sub-national constraint that coal use must not increase above present levels in three largest urban regions (the Beijing-Tianjin-Hebei Area, Yangtze River Delta, and Pearl River Delta) through 2030. Using GEOS-Chem, we project changes in the surface concentration of ozone and PM$_{2.5}$ over China and the U.S. in 2030. We find that air pollutants decrease substantially over both China and the U.S. under the national targets set by the Air Pollution Action Plan.
