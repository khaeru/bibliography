Climate policies targeting energy-related CO2 emissions, which act on a global scale over long time horizons, can result in localized, near-term reductions in both air pollution and adverse human health impacts. Focusing on China, the largest energy-using and CO2-emitting nation, we develop a cross-scale modelling approach to quantify these air quality co-benefits, and compare them to the economic costs of climate policy. We simulate the effects of an illustrative climate policy, a price on CO2 emissions. In a policy scenario consistent with China’s recent pledge to reach a peak in CO2 emissions by 2030, we project that national health co-benefits from improved air quality would partially or fully offset policy costs depending on chosen health valuation. Net health co-benefits are found to rise with increasing policy stringency.
