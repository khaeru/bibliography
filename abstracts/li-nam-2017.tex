In this study, we demonstrate the existence of bimodality in China's city-size distribution and develop an urban-growth forecast model that incorporates this bimodality. Main data for our analysis are                                                                           {\$}{\$}0.{\{}25{\}}^{\{}{\backslash}circ {\}}{\backslash}times 0.{\{}25{\}}^{\{}{\backslash}circ {\}}{\$}{\$}                                                                                    0                        .                                                                              25                                                    ∘                                                {\texttimes}                        0                        .                                                                              25                                                    ∘                                                                                                     population density grids for the past 32 years, created from China's official census data and county-level statistics. Our results show that the mixture of two Gaussian distributions outperforms unimodal distributions in explaining China's historic urban-growth patterns, suggesting that the conventional unitary urban-hierarchy assumption lacks ground in China's context. We also find that the higher-density mixture component increasingly dominates the entire distribution, and this gradual transition toward a unimodal city-size distribution is partly related to increased domestic population mobility.