Fossil-fuel subsidies contribute to the extensive growth of energy demand and the related carbon dioxide emissions in China. However, the process of energy price reform is slow, even though China faces increasing problems of energy scarcity and environmental deterioration. This paper focuses on analyzing fossil fuel subsidies in China by estimating subsidies scale and the implications for future reform. We begin by measuring fossil-fuel subsidies and the effects of subsidy removal in a systematic fashion during 2006–2010 using a price-gap approach. Results indicate that the oil price reform in 2009 significantly reduced China’s fossil-fuel subsidies and modified the subsidy structure. Fossil-fuel subsidies scale in China was 881.94 billion CNY in 2010, which was lower than the amount in 2006, equivalent to 2.59% of the GDP. The macro-economic impacts of removing fossil-fuel subsidies are then evaluated by the computable general equilibrium (CGE) model. Results demonstrate that the economic growth and employment will be negatively affected as well as energy demand, carbon dioxide and sulfur dioxide emissions. Finally, policy implications are suggested: first, risks of government pricing of energy are far from negligible; second, an acceptable macroeconomic impact is a criterion for energy price reform in China; third, the future energy policy should focus on designing transparent, targeted and efficient energy subsidies.