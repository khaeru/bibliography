Household car ownership has risen dramatically in China over the past decade. At the same time a disruptive transportation technology emerged, the electric bike (e-bike). Most studies investigating motorization in China focus on macro-level economic indicators like GDP, with few focusing on household, city-level, environmental, or geographic indicators, and none in the context of high e-bike ownership. This study examines household vehicle purchase decisions across 59 cities in China with broad geographic, environmental, and socio-economic characteristics. We focus on a subset of households who own e-bikes and rely on a telephone survey from an industry customer database. From these responses, we estimate two three-level hierarchical choice models to assess attributes that contribute to (1) recent car purchases and (2) the intention to buy a car in the near future. The results show that the models are dominated by household characteristics including household income, household size, household vehicle ownership, number of licensed drivers and duration of car ownership. Some geographic, environmental and socio-economic factors have significant influences on car purchase decisions. Only two city-level transportation variable have an effect – higher taxi density and higher bus density reducing car purchase. Cold weather, population density gross domestic product per capita positively influence car purchase, while urbanization rate reduces car purchase. Because of supply heterogeneity in the data set, described by publicly available urban transportation data, this is the first study that can include geographic and urban infrastructure differences that influence purchase choice and suggests potential region-specific policy approaches to managing car purchase may be necessary.
