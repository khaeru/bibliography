There is a growing interest in transportation pricing reforms (increased fuel taxes, efficient road and parking pricing, and distance-based vehicle insurance and registration fees) to help achieve various policy objectives including reduced traffic congestion, accidents and pollution emissions. Their effectiveness is affected by the price sensitivity of vehicle fuel consumption and travel, measured as elasticities (percentage change in consumption caused by a percentage change in price). Lower elasticities imply that price reforms are relatively ineffective in achieving objectives, high prices significantly harm consumers, and rebound effects are small so strategies that increase vehicle fuel efficiency are relatively effective at conserving fuel. Higher elasticities imply that price reforms are relatively effective in achieving objectives, consumers can easily reduce fuel consumption and vehicle travel, and rebound effects are relatively large. Some studies found that \{US\} price elasticities declined during the last quarter of the Twentieth Century but recent evidence suggests that vehicle travel has since become more price sensitive. This article examines evidence of changing vehicle fuel and travel elasticities, and discusses policy implications.