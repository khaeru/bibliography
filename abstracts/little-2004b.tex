Managerial models for practice have undergone remarkable growth in the past 50 years. My paper on decision calculus, published in 1970, was both a progress report and a prescription for improvement. This commentary describes why I wrote the paper, my perception of why it has been considered influential, and a brief overview of what has happened since. The overview starts by tracing the trends and ideas of the 1960s into the 1970s. The 1970s blend easily into the era of decision support systems (DSS). Starting late in the decade and still continuing, DSSs have evolved rapidly due to an explosion of data, increased computer power, and advances in modeling methods. I review highlights of this evolution from the point of view of managerial models, giving special emphasis to marketing, since that has been my window on this world. The good news is that more managers than ever are using models. The bad news is that many managers do not even realize they are using models! (But we should ask whether this is really bad.) The commentary concludes with thoughts about the future of managerial models. ¶ The selection of my paper as one of the most influential in the first 50 years of Management Science was a significant honor for which I am grateful. I take it as an indication by the members of INFORMS of the importance to our field of the issues addressed.
