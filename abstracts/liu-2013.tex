Transport is one of the most challenge sectors when addressing energy security and climate change due to its high reliance on oil products and lack of the alternative fuels. This paper explores the ability of three transport strategies to contribute to the development of a sustainable transport in China. With this purpose in mind, a Chinese transport model has been created and three current transport strategies which are high speed railway (HSR), urban rail transit (URT) and electric vehicle (EV) were evaluated together with a reference transport system in 2020. As conservative results, 13% of the energy saving and 12% of the CO2 emission reduction can be attained by accomplishing three strategies compared with the reference transport system. However, the energy demand of transport in 2020 with the implementation of three strategies will be about 1.7 times as much as today. The three strategies show the potential of drawing the transport demand to the more energy efficient vehicles; however, more initiatives are needed if the sustainable transport is the long term objective, such as the solutions to stabilise the private vehicle demands, to continuously improve the vehicle efficiency and to boost the alternative fuels produced from the renewable energy sources.