Vehicular evaporative emissions is an important source of volatile organic carbon (VOC), however, accurate estimation of emission amounts and scientific evaluation of control strategy for these emissions have been neglected outside of the United States. This study provides four kinds of basic emission factors: diurnal, hot soak, permeation, and refueling. Evaporative emissions from the Euro 4 vehicles (1.6 kg/year/car) are about four times those of U.S. vehicles (0.4 kg/year/car). Closing this emissions gap would have a larger impact than the progression from Euro 3 to Euro 6 tailpipe HC emission controls. Even in the first 24 h of parking, China’s current reliance upon the European 24 h diurnal standard results in 508 g/vehicle/year emissions, higher than 32 g/vehicle/year from Tier 2 vehicles. The U.S. driving cycle matches Beijing real-world conditions much better on both typical trip length and average speed than current European driving cycles. At least two requirements should be added to the Chinese emissions standards: an onboard refueling vapor recovery to force the canister to be sized sufficiently large, and a 48-h evaporation test requirement to ensure that adequate purging occurs over a shorter drive sequence.
