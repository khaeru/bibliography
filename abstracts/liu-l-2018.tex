With the accelerating process of urbanization, energy consumption and emissions of the transport sector in China have increased rapidly. In this paper, we employed the LEAP (Long-range Energy Alternatives Planning system) model to estimate the energy consumption, CO2 (carbon dioxide) and air pollutant emissions of the transport sector between 2010 and 2050 under four scenarios: Business as Usual (BAU), Energy Efficiency Improvement (EEI), Transport Mode Optimization (TMO), and Comprehensive Policy (CP). Furthermore, the intake fraction method was adopted to assess the health benefits of reducing pollutant emissions. The results showed that energy consumption will reach 509–1284 Mtce under the different scenarios by 2050. The emissions of CO2, carbon monoxide (CO), sulfur dioxide (SO2), nitrogen oxide (NOX) and particulate matter (PM10 and PM2.5) will be 2601, 173, 3.4, 24.0, 0.94 and 0.78 Mt, respectively, under the BAU scenario in 2050. Regarding health benefits, economic losses caused by mortality will be reduced by 47, 40 and 72 billion USD in 2050 under the EEI, TMO and CP scenarios, respectively, compared to those under the BAU scenario. Among the health outcomes associated with PM10, acute bronchitis exhibits the worst outcome. Considering health impacts, policy implications are suggested to reduce CO2 and pollutant emissions.