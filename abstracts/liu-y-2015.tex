China's urbanization is rapidly proceeding while bringing marked impact on the energy consumption. This paper contributes to current literature by using an improved spatial econometric model Stochastic Impacts by Regression on Population, Affluence and Technology (STIRPAT) to investigate the effect of urbanization on energy consumption for China's regions. The impact of urbanization is further disaggregated into direct and indirect impacts from the perspective of spatial econometrics. Results show that, if the urbanization level increases by 1.0%, energy consumption levels will correspondingly decrease by 0.089%. The spatial spillover effect on adjacent areas is positive: a 1.0% increase in one specific area's urbanization level leads to a 0.136% increase in an adjacent area's energy consumption. It is suggested that, it is essential to devote major efforts to energy conservation, in the process of improving the new-type urbanization. In particular, disaggregating China's targets for controlling energy consumption into smaller targets for different regions will contribute to controlling the growth of energy consumption.