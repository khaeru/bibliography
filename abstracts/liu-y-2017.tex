The driving restriction policy has been implemented to alleviate traffic congestion and air pollution in Beijing. Because of the traditionally superstitious aversion to the number four and preference for six and eight, Chinese people always consciously avoid the former while tend to choose the latter two for a vehicle license plate. As a result, there will be a significantly variation in circulating cars on roads between days with different numbers as the last digit of the license plates restricted. Leveraging this exogenous variation and daily data of 2009, we applied the generalized additive model to explore the association of driving restrictions and daily hospital admissions for respiratory disease in Beijing. Regression results revealed that banning 4 days with the number four experienced a 2.24 {\%} [95 {\%} confidence interval (CI) 1.73--2.77 {\%}, p < 0.01] higher daily hospital admissions for respiratory disease than other restricting days. The health effect was significantly stronger in cold season when heating service is provided than in warm season. Besides, females and residents aged ≥65 years old benefitted more from this environmental policy. Our findings indicate that Beijing's driving restriction policy routinely restricting 20 {\%} circulating cars on road every day may have positive effects on the improvement of public health.