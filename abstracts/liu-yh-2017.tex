Exploring vehicle emission trends within and outside the Pearl River Delta (PRD) region during a long period was scientific and practical, for the economic rapid unbalanced development, continuous implements of severe reducing vehicle emissions measures in Guangdong province. Multi-year inventories of vehicle emissions from 1994 to 2014 were estimated based on the emissions factors of different emissions standards and vehicle kilometers travelled for all types of vehicles. The trends and characteristics of the emissions of carbon monoxide (CO), volatile organic compounds (VOCs), nitrogen oxides (NOx), fine particulate matter (PM2.5) and course particulate matter (PM10) were then analyzed within and outside the PRD region. In the above two regions, the total amount of the five pollutant emissions varied greatly with gross domestic product (GDP) from 1994 to 2014, showing the overall performance of the first increasing up to 1.6–3.0 times before 2002, and then decreasing. However, the five pollutant emissions in the PRD region were 2.4–3.3 times more than in the non-PRD region. In both regions, light passenger cars and motorcycles were the main contributors to CO and VOC emissions (65%–80%), and heavy duty trucks and passenger cars were the main contributors to NOx, PM2.5 and PM10 emissions (around 42%–50%). Moreover, compared to CO and VOCs emissions, the changes in the contribution of every vehicles type to NOx, PM2.5 and PM10 emissions were more obvious, and coincided with the implementation time of emission and fuel standards in the non-PRD region. It was noted that CO and VOC emission variations was correlated closely with the population of yellow-label light passenger cars and motorcycles, whereas those of NOx and PM2.5 was coincided that of yellow-label heavy passenger cars and trucks.