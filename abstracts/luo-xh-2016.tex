Interprovincial migration flows involve substantial relocation of people and productive activity, with implications for regional energy use and greenhouse gas emissions. In China, these flows are not explicitly considered when setting energy and environmental targets for provinces, and their potential impact on the effectiveness of policy alternatives is ignored. We analyze how migration affects outcomes under energy intensity targets and energy caps. While both policies are part of the nation's Twelfth Five Year Plan (2011–2015) and imposed at the provincial level, only the intensity targets are binding at present. We estimate a migration model, integrate it into a general equilibrium model that resolves each province in China, and simulate the effect of migration on energy use and economic activity. We find that although both types of policies are affected by uncertain migration flows, energy intensity targets (energy use indexed to economic output) are more robust than absolute caps. They are also more cost effective, placing less burden on the relatively clean in migration provinces. Our findings also underscore the value of moving from provincial targets to an integrated national trading system targeting emissions of energy-related CO2, given that the choice of abatement strategies will adjust endogenously to labor relocation.
