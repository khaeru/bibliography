India is in the course of an economic transition. The economic growth nurtured the life in the cities and cities have become a major livelihood destination for everyone. This migration of people contributed to the increased urbanization of Indian cities. The booming economy fostered the well-being and shaped the lifestyle of people in such a way that the dependency on private vehicle has become an unavoidable affair. Along with population growth, the increased vehicle ownership gave rise to overall spurt in travel demand. But the supply side lagged behind the demand adding to many of the transport related externalities such as accidents, congestion, pollution, inequity etc. The importance of sustainability is understood in the current urban transport scenario leading to the development and promotion of sustainable transport polices. The core agenda of these polices is to target the travel behavior of people and change the way they travel by creating a different travel environment. However, the impacts of many such policies are either unknown or complex. Hence, before adopting and implementing such policies, it is important for the decision makers to be aware of the impacts of them. The role of travel demand models comes here as they predict the future travel demand under different policy scenarios. This paper reviews the ability of travel demand models applied in India in analyzing the sustainable transport policies. The study found that the conventional model system in India, which is trip based four step aggregate methodology, is inadequate in analyzing the sustainable transport policies. A review of alternative approach, known as activity based travel demand modeling found that they are capable of handling such policies better than conventional models and are assistive to the decision makers in arriving at right mix of polices specific to the situations. Since there is no operational activity based travel demand model system developed in India, the study at the end envisaged a conceptual framework of an integrated activity based travel demand model based on the requirements identified from the review. This can potentially replace the existing travel demand models and can be used for planning applications once the modification & validation have been done according to the existing activity-travel behavior of individuals.