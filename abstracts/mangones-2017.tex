Recent and continuing advances in active safety vehicle technology have created new possibilities for improving road safety. Crash-avoidance systems with the capability to warn and assist drivers are now available in most automotive market in developed countries. Despite this availability, investment in high-technology transit buses to prevent fatalities and injuries has been very slow. In this paper, we examine the safety benefits of using forward- and side-collision warning systems and active collision-avoidance systems in transit buses in New York City and Bogota, Colombia. Using historical data, we develop a transportation risk profile for each city by type of user (driver, passenger, pedestrian, and bicyclist) and crash severity. Because there is no historical data on the effectiveness of crash avoidance systems on buses in the U.S., we surveyed 12 leading experts on autonomous and connected vehicles to assess the potential reduction in injuries and fatalities in road crashes that involve buses. We report on the agreements and disagreements of expert’s judgments and contrast their judgment with the breakeven combination of risk reduction needed to overcome the cost of providing crash avoidance technology in the bus fleet. Additionally, we perform a benefit-cost analysis under uncertainty using Monte Carlo simulation to compute distributions of benefit-cost-ratio. The benefit-cost analysis reveals that implementing any of the technologies in {NYC} is economically justifiable. In Bogota, even though fatality and injury risks are higher, statistical valuation of lives and injuries are much lower. As a consequence, policy makers are likely to reject the investment in the technologies.
