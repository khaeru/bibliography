This paper reports on a study of current practice in policy transfer, and ways in which its effectiveness can be increased. A literature review identifies important factors in examining the transfer of policies. Results of interviews in eleven cities in Northern Europe and North America investigate these factors further. The principal motivations for policy transfer were strategic need and curiosity. Local officials and politicians dominated the process of initiating policy transfer, and local officials were also the leading players in transferring experience. A range of information sources are used in the search process but human interaction was the most important source of learning for two main reasons. First, there is too much information available through the Internet and the search techniques are not seen to be wholly effective in identifying the necessary information. Secondly, the information available on websites, portals and even good practice guides is not seen to be of mixed quality with risks of focussing only on successful implementation and therefore subject to some bias. Officials therefore rely on their trusted networks of peers for lessons as here they can access the ‘real implementation’ story and the unwritten lessons. Organisations which have a culture that is supportive of learning from elsewhere had strong and broad networks of external contacts and resourced their development whilst others are more insular or inward looking and reluctant to invest in policy lessons from elsewhere. Solutions to the problems identified in the evidence base are proposed. City to city policy transfer is a very active process in the field of transport. Not enough is yet understood about its benefits or the conditions under which it is most effective. Such understandings should help to promote and accelerate the uptake of effective and well matched policies.
