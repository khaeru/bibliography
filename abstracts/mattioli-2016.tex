This paper identifies three main understandings of the notion of ‘car dependence’ in transport research: a micro-social understanding (dependence as an attribute of individuals), a macro approach (attribute of societies or local areas as whole), and a meso-level understanding, where it refers to trips – or rather to the activities that people travel to undertake. While the first two approaches have been dominant, this paper further develops the third, addressing questions as to whether and why certain activities are inherently more difficult to switch away from the car. At the theoretical level, it builds on theories of social practice to put forward the notion of ‘car dependent practices’. At the empirical level, it demonstrates that the application of sequence pattern mining techniques to time use data allows the identification of car and mobility intensive activities, arguably representing the trace of car dependent practices. Overall, the findings of this mining exercise suggest that the emphasis of existing literature on escorting children, shopping and carrying heavy goods as car dependent trip purposes is not misplaced. Our analysis adds to this knowledge by contextualising the information by providing detailed quantitative analysis of a larger, richer set of activities hitherto overlooked in transport policy. The article concludes by illustrating the policy implications of the approach adopted and the findings generated, discussing possible strategies to steer practices in a more sustainable direction by creating material alternatives to the ‘cargo function’ of car travel.