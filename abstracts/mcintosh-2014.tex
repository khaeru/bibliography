Car dependence is in decline in most developed cities, but its cause is still unclear as cities struggle with priorities in urban form and transport infrastructure. This paper draws conclusions from analysis of data in 26 cities over the last 40years of the 20th century. Statistical modelling techniques are applied to urban transport and urban form data, while examining the influence of region, city archetype and individual fixed effects. Structural equation modelling is employed to address causation and understand the direct and indirect effects of selected parameters on per capita vehicle kilometres travelled (VKT). Findings suggest that, while location effects are important, transit service levels and urban density play a significant part in determining urban car use per capita, and causality does flow from these factors towards a city’s levels of private vehicle travel as well as the level of the provision of road capacity.