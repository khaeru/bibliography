This paper examines the effect of economic development on the demand for private motor vehicles for a panel of 28 countries. Utilising the concept of the user cost of capital and the notion that the demand for cars can become saturated, the authors develop a model of the relationship between economic development and per capita private car ownership. They find that saturation levels vary across countries, and that user costs are a significant factor in the evolution of vehicle stocks. Forecasts are generated for each of the countries in the sample, and the implications for future energy-related issues are discussed.
