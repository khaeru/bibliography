An integrated life cycle approach framework, including material flow analysis (MFA), Cumulative Energy Demand (CED), exergy analysis (EXA), Emergy Assessment (EMA), and emissions (EMI) has been constructed and applied to examine the energy efficiency of high speed urban bus transportation systems compared to conventional bus transport in the city of Xiamen, Fujian province, China. This paper explores the consistency of the results achieved by means of several evaluation methods, and explores the sustainability of innovation in urban public transportation systems. The case study dealt with in this paper is a Bus Rapid Transit (BRT) system compared to Normal Bus Transit (NBT). All the analyses have been performed based on a common yearly database of natural resources, material, labor, energy and fuel input flows used in all life cycle phases (resource extraction, processing and manufacturing, use and end of life) of the infrastructure, vehicle and vehicle fuel. Cumulative energy, material and environmental support demands of transport are accounted for. Selected pressure indicators are compared to yield a comprehensive picture of the public transportation system. Results show that Bus Rapid Transit system (BRT) shows much better energy and environmental performance than NBT, as indicated by the set of sustainability indicators calculated by means of our integrated approach. This is because of the higher efficiency of such modality (less affected by traffic, higher vehicle occupancy, suitability for large distance transportation). The study suggests that the higher economic and resource investments performed in order to provide dedicated roads, more modern transport technology and higher speed, translated into a better use of resources and lower environmental pressure, also because of the attraction of an increased number of passengers, who would have otherwise used car transportation modalities. This study also provides a clear evidence that more than one criterion is needed to address a fully reliable and sustainable urban transportation policy. An integrated approach is therefore suggested to support decision making in the presence of complex systems and different kinds of concerns to be taken into proper account.