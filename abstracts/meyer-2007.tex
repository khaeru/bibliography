The paper provides global regionalized projections of passenger car demand, use and associated CO$_2$ emissions from 11 world regions. The study is based on empirical data that have been originally collated from international sources for the purpose of modeling region-specific car stock demand. Derived demands serve as indicator of car related fuel consumption and associated CO$_2$ emissions, which are calculated on the basis of behavioral and technological scenarios. The obtained CO$_2$ emission paths are sectoral baseline scenarios that identify region-specific potentials of growth in car induced CO$_2$ emissions assuming that current trends continue to prevail. The study adopts a multi-model approach to car demand by applying two methodologies rooted in the economics of consumption: utility maximization and single equation models. The utility maximization method for modeling car demand is driven by the preferences of the representative consumers of each world region, subject to exogenous price and income trajectories. The latter is adopted from an optimal growth model. This is a novel approach to projecting global regionalized sectoral car demands. The study is complemented by the application of single equation income–consumption models based on logistical Gompertz functions and non-linear regression techniques to compare model results.
