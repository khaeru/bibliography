We assess the economic value of life-cycle air emissions and oil consumption from conventional vehicles, hybrid-electric vehicles (HEVs), plug-in hybrid-electric vehicles (PHEVs), and battery electric vehicles in the US. We find that plug-in vehicles may reduce or increase externality costs relative to grid-independent HEVs, depending largely on greenhouse gas and SO$_2$ emissions produced during vehicle charging and battery manufacturing. However, even if future marginal damages from emissions of battery and electricity production drop dramatically, the damage reduction potential of plug-in vehicles remains small compared to ownership cost. As such, to offer a socially efficient approach to emissions and oil consumption reduction, lifetime cost of plug-in vehicles must be competitive with HEVs. Current subsidies intended to encourage sales of plug-in vehicles with large capacity battery packs exceed our externality estimates considerably, and taxes that optimally correct for externality damages would not close the gap in ownership cost. In contrast, HEVs and PHEVs with small battery packs reduce externality damages at low (or no) additional cost over their lifetime. Although large battery packs allow vehicles to travel longer distances using electricity instead of gasoline, large packs are more expensive, heavier, and more emissions intensive to produce, with lower utilization factors, greater charging infrastructure requirements, and life-cycle implications that are more sensitive to uncertain, time-sensitive, and location-specific factors. To reduce air emission and oil dependency impacts from passenger vehicles, strategies to promote adoption of HEVs and PHEVs with small battery packs offer more social benefits per dollar spent.
