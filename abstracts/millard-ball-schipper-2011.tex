Projections of energy use and greenhouse gas emissions for industrialized countries typically show continued growth in vehicle ownership, vehicle use and overall travel demand. This growth represents a continuation of trends from the 1970s through the early 2000s. This paper presents a descriptive analysis of cross‐national passenger transport trends in eight industrialized countries, providing evidence to suggest that these trends may have halted. Through decomposing passenger transport energy use into activity, modal structure and modal energy intensity, we show that increases in total activity (passenger travel) have been the driving force behind increased energy use, offset somewhat by declining energy intensity. We show that total activity growth has halted relative to GDP in recent years in the eight countries examined. If these trends continue, it is possible that an accelerated decline in the energy intensity of car travel; stagnation in total travel per capita; some shifts back to rail and bus modes; and at least somewhat less carbon per unit of energy could leave the absolute levels of emissions in 2020 or 2030 lower than today. 
