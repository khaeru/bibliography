This paper assesses comparable urban transport scenarios for China and India. The assessment methodology uses AIM/End-use model with a detailed characterization of technologies to analyze two scenarios for India and China till the year 2050. The first scenario assumes continuation and enhancement, in both countries, of policies under a typical business-as-usual dynamics, like constructing metros, implementing national fuel economy standards, promoting alternate fuel vehicles and implementing national air quality standards. The alternative, low carbon scenario assumes application, in both countries, of globally envisaged measures like fuel economy standards as well as imposition of carbon price derived from a global integrated assessment modeling exercise aiming to achieve global 2 °C temperature stabilization target. The modeling results for both countries show that decarbonizing transport sector shall need a wide array of measures including fuel economy, low carbon fuel mix including low carbon electricity supply. The comparison of China and India results provides important insights and lessons from their similarities and differences in the choice of urban transport options. India can benefit from China’s experiences as it lags China in urbanization and income. Modeling assessments show that both nations can contribute to, as well as benefit by aligning their transport plans with global climate stabilization regime.