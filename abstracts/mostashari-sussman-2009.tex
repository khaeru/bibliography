Standard systems engineering processes, such as the ANSI/EIA 632 or the ISO/IEC 15288 process standards, are primarily geared towards systems with homogenous stakeholders and few tightly coupled non-linear interactions within and between physical subsystems. When dealing with Complex, Large-scale, Interconnected, Open, Sociotechnological (CLIOS) systems such as the national power grid, regional transportation systems, the world wide web or the air traffic control system or other systems with wide-ranging social and environmental impact, there is a need for a new framework that takes into consideration the physical and institutional system complexities and their respective interactions in an iterative and adaptive manner. This paper presents a 12-step framework for concurrent analysis, design and management of coupled complex technological and institutional systems in the face of uncertainty. The framework can be used to analyze a CLIOS system’s underlying structure and behavior, to explore different design options in the face of uncertainty, and to identify and deploy strategic alternatives for improving the system’s performance.
