When explanatory variable data in a regression model are drawn from a population with grouped structure, the regression errors are often correlated within groups. Error component and random coefficient regression models are considered as models of the intraclass correlation. This paper analyzes several empirical examples to investigate the applicability of random effects models and the consequences of inappropriately using ordinary least squares (OLS) estimation in the presence of random group effects. The principal findings are that the assumption of independent errors is usually incorrect and the unadjusted OLS standard errors often have a substantial downward bias, suggesting a considerable danger of spurious regression.