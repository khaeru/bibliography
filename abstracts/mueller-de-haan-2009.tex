This article presents an agent-based microsimulation capable of forecasting the effects of policy levers that influence individual choices of new passenger cars. The fundamental decision-making units are households distinguished by sociodemographic characteristics and car ownership. A two-stage model of individual decision processes is employed. In the first stage, individual choice sets are constructed using simple, non-compensatory rules that are based on previously owned cars. Second, decision makers evaluate alternatives in their individual choice set using a multi-attributive weighting rule. The attribute weights are based on a multinomial logit model for cross-country policy analysis in European countries. Additionally, prospect theory and the notion of mental accounting are used to model the perception of monetary values. The microsimulation forecasts actual market observations with high accuracy, both on the level of aggregate market characteristics as well as on a highly resolved level of distributions of market shares. The presented approach is useful for the assessment of policies that influence individual purchase decisions of new passenger cars; it allows accounting for a highly resolved car fleet and differentiated consumer segments. As a result, the complexity of incentive schemes can be represented and detailed structural changes can be investigated.