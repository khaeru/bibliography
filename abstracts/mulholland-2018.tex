Road freight transportation is a key enabler of global economic activity while also a central consumer of fossil fuels, which presents a challenge in realising a low-carbon future. To identify feasible decarbonisation solutions, we first assess significant drivers of activity in the road freight sector. We then use these drivers to project road freight service demand, vehicle stock, mileage, sales, final energy demand, and well-to-wheel GHG emissions using the IEA’s Mobility Model (MoMo) under two scenarios – the first incorporating the policy ambition of the Nationally Determined Contributions pledged at COP21, and the second extending ambitions to emission reductions that are in line with limiting global temperature rise to 1.75 degrees. In the former scenario, road freight well-to-wheel GHG emissions increase by 56% between 2015 and 2050, while in the latter, sectoral emissions are reduced by 60% over the same period, reflecting our assessment of the threshold of emission reductions potential. This reduction is catalysed by policy efforts including fuel economy regulations, carbon taxes on transport fuels, differentiated distance-based pricing, widespread data-sharing and collaboration across the supply chain as enabled by digital technologies, and sustained investment in ultra-low and zero-carbon infrastructure and research development and deployment.