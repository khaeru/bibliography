After 50 years of development, building and solving mixed-complementarity problems (MCPs) have become commonplace for policy models that analyze markets. These models are used to develop policies that reshape markets, or introduce markets that replace other organizational forms. In this tutorial, we give some background on building economic equilibrium models, starting with the use of linear programming, and show how MCPs can be used to answer policy questions that require manipulation of the solutions to linear programs of economic sectors. We illustrate the use of MCPs using examples from King Abdullah Petroleum Studies and Research Center projects, including a model of domestic energy markets in Saudi Arabia, which is in the process of changing some of its pricing policies and market regulations.