This study explores whether China's population distribution is excessively biased toward large cities or coastal regions. The test is based on a fixed effects model estimated from a 5-year panel dataset for 101 countries, and two spatial inequality measures are computed from                                                                           {\$}{\$}0.25^{\{}{\backslash}circ {\}}{\backslash}times 0.25^{\{}{\backslash}circ {\}}{\$}{\$}                                                                                    0                        .                                                  25                          ∘                                                {\texttimes}                        0                        .                                                  25                          ∘                                                                                                     population grids for a parallel cross-country comparison. The results show that the spatial Gini coefficient for China does not deviate from a general trend, while Moran's I index is biased upward. This suggests that the spatial inequality of China's population distribution tends to be more obvious at the regional level than at the city level.