This paper aims to compare the descriptive and predictive power of two classes of statistical estimation models for multimodal network flows, viz. the family of discrete choice models (i.e., logit and probit models) and the neural network model. The application concerns a large dataset on interregional European freight flows for two commodity categories (food and chemicals). After a concise exposition of policy issues, methodological and modelling questions and the database, a variety of experiments is carried out. The results show that in general the predictive potential of neural network models is higher than that of discrete choice analysis. The statistical results are also used to investigate the implications of various road charge systems (e.g., eco-taxes) on specific spatial segments of the European road network.