Over the past four decades, bicycle ownership has been documented in various countries but not globally analyzed. This paper presents an effort to fill this gap by tracking household bicycle possession. First, we gather survey data from 150 countries and extract percentage household bicycle ownership values. Performing cluster analysis, we determined four groups with the weighted mean percentage ownership ranging from 20\% to 81\%. Generally, bicycle ownership was highest in Northern Europe and lowest in West, Central and North Africa, and Central Asia. We determine worldwide household ownership patterns and demonstrate a basis for understanding the global impact of cycling as a sustainable transit mode. Furthermore, we find a lower-bound estimate of the number of bicycles available to the world׳s households. We establish that at the global level 42\% of households own at least one bicycle, and thus there are at least 580 million bicycles in private household ownership. Our data are publicly available and amenable for future analyses.
