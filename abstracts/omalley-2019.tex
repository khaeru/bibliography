Energy systems are transforming, based largely on the drive to decarbonize. Energy system decarbonization can be achieved in many ways including increasing renewable energy, carbon capture, increased nuclear and probably the most effective, reduction in energy use and waste.
An integrated energy systems approach is fundamental for effective decarbonization and energy system models are being developed and deployed at a rapid pace in order to inform and enable decarbonization.
However, these models are limited, limiting and limitless all at the same time.
The seminar will first introduce the concept of energy systems integration.
I will then share aspects of my work on the integration of variable renewable energy into electricity grids to illustrate its central role in decarbonization and the critical role of modelling.
These contributions include insights into variable renewable energy characteristics (i.e. generator technology and resource variability and uncertainty) and the consequential development of new methods and models for the operation and planning of electricity grids.
The advantages of coupling electricity to other energy vectors (e.g. heat) and the electrification of large parts of the economy will be illustrated along with the need for more comprehensive models.
Finally, some research challenges in particular with respect to models will be proposed and the need for collaboration between academia and industry, across disciplines and internationally in overcoming these challenges will be emphasized.
