Long-term scenarios play an important role in research on global environmental change. The climate change research community is developing new scenarios integrating future changes in climate and society to investigate climate impacts as well as options for mitigation and adaptation. One component of these new scenarios is a set of alternative futures of societal development known as the shared socioeconomic pathways (SSPs). The conceptual framework for the design and use of the SSPs calls for the development of global pathways describing the future evolution of key aspects of society that would together imply a range of challenges for mitigating and adapting to climate change. Here we present one component of these pathways: the SSP narratives, a set of five qualitative descriptions of future changes in demographics, human development, economy and lifestyle, policies and institutions, technology, and environment and natural resources. We describe the methods used to develop the narratives as well as how these pathways are hypothesized to produce particular combinations of challenges to mitigation and adaptation. Development of the narratives drew on expert opinion to (1) identify key determinants of these challenges that were essential to incorporate in the narratives and (2) combine these elements in the narratives in a manner consistent with scholarship on their inter-relationships. The narratives are intended as a description of plausible future conditions at the level of large world regions that can serve as a basis for integrated scenarios of emissions and land use, as well as climate impact, adaptation and vulnerability analyses.