South Africa is facing the triple challenge of (a) fuelling economic development and meeting the growing energy demand; (b) increasing the reliability of the electricity system; and (c) ensuring that domestic greenhouse gas emissions peak no later than 2030 to meet its nationally determined contributions (NDC) under the 2015 Paris Agreement. Recently discovered domestic shale gas reserves are being considered as a potential new energy source to provide clean, reliable and cheap electricity while mitigating greenhouse gas emissions relative to the dominant coal sector. In order to determine if shale gas can play a viable role in solving South Africa's energy trilemma, we apply a country-level version of the integrated assessment model MESSAGEix to analyze and quantify the interdependencies between shale gas, the energy system and South Africa's greenhouse gas emissions trajectory. Our results indicate that shale gas extraction costs must be below 3\,USD/GJ for this energy source to reach a significant share in the fuel mix; this is well below current cost estimates. If, however, low-cost shale gas is available, both coal and low-carbon sources are replaced by natural gas. Whether carbon dioxide emissions increase or decrease as a result depends on the stringency of the climate change mitigation policy in place: without carbon pricing, natural gas replaces coal and mitigates harmful emissions; under high carbon prices, power generation from coal is phased out in any case, and natural gas competes with zero-carbon renewables, leading to an increase of emissions compared to a no-shale scenario.