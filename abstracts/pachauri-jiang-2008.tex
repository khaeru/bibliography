Both India and China are countries in energy transition. This paper compares the household energy transitions in these nations through the analysis of both aggregate statistics and nationally representative household surveys. The two countries differ sharply in several respects. Residential energy consumption in China is twice that in India, in aggregate terms. In addition, Chinese households have almost universal access to electricity, while in India almost half of rural households and 10% of urban households still lack access. On aggregate, urban households in China also derive a larger share of their total energy from liquid fuels and grids (77%) as compared to urban Indian households (65%). Yet, at every income level, Indians derive a slightly larger fraction of their total household energy needs from liquid and grid sources of energy than Chinese with comparable incomes. Despite these differences, trends in energy use and the factors influencing a transition to modern energy in both nations are similar. Compared with rural households, urban households in both nations consume a disproportionately large share of commercial energy and are much further along in the transition to modern energy. However, total energy consumption in rural households exceeds that in urban households, because of a continued dependence on inefficient solid fuels, which contribute to over 85% of rural household energy needs in both countries. In addition to urbanisation, key drivers of the transition in both nations include income, energy prices, energy access and local fuel availability.