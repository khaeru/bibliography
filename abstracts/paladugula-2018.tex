This paper focuses on comparing the framework and projections of energy consumption and emissions from India's transportation sector up to 2050. To understand the role of road transport in energy demand and emissions, five modeling teams developed baseline projections for India's transportation sector as part of inter-model comparison exercise under the Sustainable Growth Working Group (SGWG) of the US-India Energy Dialog. Based on modeling results, we explore the developments in India's passenger and freight road transport, including changes in the modal shift and the resulting changes in energy consumption, carbon dioxide (CO2) and particulate matter (PM2.5) emissions. We find significant differences in the base-year data and parameters for future projections, namely energy consumption by transport in general and by mode, service demand for passenger and freight transport. Variation in modeling assumptions across modeling teams reflects the difference in opinion among the different modeling teams which in turn reflects the underlying uncertainty with respect to key assumptions. We have identified several important data gaps in our knowledge about the development of the transportation sector in India. The results of this inter-model study can be used by Indian policy makers to set quantified targets in emission reductions from the transportation sector.