The European Union (EU) recently adopted CO2 emissions mandates for new passenger cars, requiring steady reductions to 95 gCO2/km in 2021. We use a multi-sector computable general equilibrium (CGE) model, which includes a private transportation sector with an empirically-based parameterization of the relationship between income growth and demand for vehicle miles traveled. The model also includes representation of fleet turnover, and opportunities for fuel use and emissions abatement, including representation of electric vehicles. We analyze the impact of the mandates on oil demand, CO2 emissions, and economic welfare, and compare the results to an emission trading scenario that achieves identical emissions reductions. We find that vehicle emission standards reduce CO2 emissions from transportation by about 50 MtCO2 and lower the oil expenditures by about {\texteuro}6 billion, but at a net added cost of {\texteuro}12 billion in 2020. Tightening CO2 standards further after 2021 would cost the EU economy an additional {\texteuro}24--63 billion in 2025, compared with an emission trading system that achieves the same economy-wide CO2 reduction. We offer a discussion of the design features for incorporating transport into the emission trading system.