Since the mid-2000s, public bikesharing (also known as “bike hire”) has developed and spread into a new form of mobility in cities across the globe. This paper presents an analysis of the recent increase in the number of public bikesharing systems. Bikesharing is the shared use of a bicycle fleet, which is accessible to the public and serves as a form of public transportation. The initial system designs were pioneered in Europe and, after a series of technological innovations, appear to have matured into a system experiencing widespread adoption. There are also signs that the policy of public bikesharing systems is transferable and is being adopted in other contexts outside Europe. In public policy, the technologies that are transferred can be policies, technologies, ideals or systems. This paper seeks to describe the nature of these systems, how they have spread in time and space, how they have matured in different contexts, and why they have been adopted. Researchers provide an analysis from Europe and North America. The analysis draws on published data sources, a survey of 19 systems, and interviews with 12 decision-makers in Europe and 14 decision-makers in North America. The data are examined through the lens of diffusion theory, which allows for comparison of the adoption process in different contexts. A mixture of quantitative and qualitative analyses is used to explore the reasons for adoption decisions in different cities. The paper concludes that Europe is still in a major adoption process with new systems emerging and growth in some existing systems, although some geographic areas have adopted alternative solutions. Private sector operators have also been important entrepreneurs in a European context, which has accelerated the uptake of these systems. In North America, the adoption process is at an earlier stage and is gaining momentum, but signs also suggest the growing importance of entrepreneurs in North America with respect to technology and business models. There is evidence to suggest that the policy adoption processes have been inspired by successful systems in Paris, Lyon, Montreal, and Washington, DC, for instance, and that diffusion theory could be useful in understanding public bikesharing policy adoption in a global context.