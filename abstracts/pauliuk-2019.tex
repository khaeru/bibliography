Until this day, data in industrial ecology (IE) have been commonly seen as existing within the domain of particular methods or models, such as input–output, life cycle assessment, urban metabolism, or material flow analysis data. This artificial division of data into methods contradicts the common phenomena described by those data: the objects and processes in the industrial system, or socioeconomic metabolism (SEM). A consequence of this scattered organization of related data across methods is that IE researchers and consultants spend too much time searching for and reformatting data from diverse and incoherent sources, time that could be invested into quality control and analysis of model results instead. This article outlines a solution to two major barriers to data exchange within IE: (a) the lack of a generic structure for IE data and (b) the lack of a bespoke platform to exchange IE datasets. We present a general data model for SEM that can be used to structure all data that can be located in the industrial system, including process descriptions, product descriptions, stocks, flows, and coefficients of all kind. We describe a relational database built on the general data model and a user interface to it, both of which are open source and can be implemented by individual researchers, groups, institutions, or the entire community. In the latter case, one could speak of an IE data commons (IEDC), and we unveil an IEDC prototype containing a diverse set of datasets from the literature.
