Nationally determined contributions (NDCs) were key to reaching the Paris Agreement and will be instrumental in implementing it. Research was quick to identify the `headline numbers' of NDCs: if these climate action plans were fully implemented, global mean warming by 2100 would be reduced from approximately 3.6 to 2.7{\textdegree}C above pre-industrial levels (H{\"o}hne et al. Climate Pol 17:1--17, 2016; Rogelj et al. Nature 534:631--639, 2016). However, beyond these headline mitigation numbers, NDCs are more difficult to analyse and compare. UN climate negotiations have so far provided limited guidance on NDC formulation, which has resulted in varying scopes and contents of NDCs, often lacking details concerning ambitions. If NDCs are to become the long-term instrument for international cooperation, negotiation, and ratcheting up of ambitions to address climate change, then they need to become more transparent and comparable, both with respect to mitigation goals, and to issues such as adaptation, finance, and the way in which NDCs are aligned with national policies. Our analysis of INDCs and NDCs (Once a party ratifies the Paris Agreement, it is invited to turn its Intended Nationally Determined Contribution (INDC) into an NDC. We refer to results from our INDC analysis rather than our NDC analysis in this commentary unless otherwise stated.) shows that they omit important mitigation sectors, do not adequately provide details on costs and financing of implementation, and are poorly designed to meet assessment and review needs.