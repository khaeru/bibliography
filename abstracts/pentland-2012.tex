Why do some teams consistently deliver high performance while other, seemingly identical teams struggle? Led by Sandy Pentland, researchers at {MIT}'s Human Dynamics Laboratory set out to solve that puzzle. Hoping to decode the "It factor" that made groups click, they equipped teams from a broad variety of projects and industries (comprising 2,500 individuals in total) with wearable electronic sensors that collected data on their social behavior for weeks at a time. With remarkable consistency, the data showed that the most important predictor of a team's success was its communication patterns. Those patterns were as significant as all other factors-intelligence, personality, talent-combined. In fact, the researchers could foretell which teams would outperform simply by looking at the data on their communication, without even meeting their members. In this article Pentland shares the secrets of his findings and shows how anyone can engineer a great team. He has identified three key communication dynamics that affect performance: energy,engagement, and exploration. Drawing from the data, he has precisely quantified the ideal team patterns for each. Even more significant, he has seen that when teams map their own communication behavior over time and then make adjustments that move it closer to the ideal, they can dramatically improve their performance.
