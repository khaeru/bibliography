We present a unique and transparent approach for incorporating social influence effects into global integrated assessment models used to analyse climate change mitigation. We draw conceptually on Rogers (2003) diffusion of innovations, introducing heterogeneous and interconnected consumers who vary in their aversion to new technologies. Focussing on vehicle choice, we conduct novel empirical research to parameterise consumer risk aversion and how this is shaped by social and cultural influences. We find robust evidence for social influence effects, and variation between countries as a function of cultural differences. We then formulate an approach to modelling social influence which is implementable in both simulation and optimisation-type models. We use two global integrated assessment models (IMAGE and MESSAGE) to analyse four scenarios that introduce social influence and cultural differences between regions. These scenarios allow us to explore the interactions between consumer preferences and social influence. We find that incorporating social influence effects into global models accelerates the early deployment of electric vehicles and stimulates more widespread deployment across adopter groups. Incorporating cultural variation leads to significant differences in deployment between culturally divergent regions such as the USA and China. Our analysis significantly extends the ability of global integrated assessment models to provide policy-relevant analysis grounded in real world processes.
