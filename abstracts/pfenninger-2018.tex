The global energy system is undergoing a major transition, and in energy planning and decision-making across governments, industry and academia, models play a crucial role. Because of their policy relevance and contested nature, the transparency and open availability of energy models and data are of particular importance. Here we provide a practical how-to guide based on the collective experience of members of the Open Energy Modelling Initiative (Openmod). We discuss key steps to consider when opening code and data, including determining intellectual property ownership, choosing a licence and appropriate modelling languages, distributing code and data, and providing support and building communities. After illustrating these decisions with examples and lessons learned from the community, we conclude that even though individual researchers' choices are important, institutional changes are still also necessary for more openness and transparency in energy research.
