This paper presents the results of the cross-validation of a multivariate logistic regression model using remote sensing data and GIS for landslide hazard analysis on the Penang, Cameron, and Selangor areas in Malaysia. Landslide locations in the study areas were identified by interpreting aerial photographs and satellite images, supported by field surveys. SPOT 5 and Landsat TM satellite imagery were used to map landcover and vegetation index, respectively. Maps of topography, soil type, lineaments and land cover were constructed from the spatial datasets. Ten factors which influence landslide occurrence, i.e., slope, aspect, curvature, distance from drainage, lithology, distance from lineaments, soil type, landcover, rainfall precipitation, and normalized difference vegetation index (ndvi), were extracted from the spatial database and the logistic regression coefficient of each factor was computed. Then the landslide hazard was analysed using the multivariate logistic regression coefficients derived not only from the data for the respective area but also using the logistic regression coefficients calculated from each of the other two areas (nine hazard maps in all) as a cross-validation of the model. For verification of the model, the results of the analyses were then compared with the field-verified landslide locations. Among the three cases of the application of logistic regression coefficient in the same study area, the case of Selangor based on the Selangor logistic regression coefficients showed the highest accuracy (94%), where as Penang based on the Penang coefficients showed the lowest accuracy (86%). Similarly, among the six cases from the cross application of logistic regression coefficient in other two areas, the case of Selangor based on logistic coefficient of Cameron showed highest (90%) prediction accuracy where as the case of Penang based on the Selangor logistic regression coefficients showed the lowest accuracy (79%). Qualitatively, the cross application model yields reasonable results which can be used for preliminary landslide hazard mapping.