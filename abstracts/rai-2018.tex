 AbstractSeveral indicators have been established to monitor and evaluate the sustainability of cities. Logistics and related transportation activities are underrepresented in these established frameworks despite the substantial negative impact of urban freight transport (UFT) on the environment, society and economy. The result is the lack of an understanding of freight flows’ impact on the liveability of cities. This paper fills this gap by introducing a comprehensive set of freight transport related indicators with an operational target in policy support and urban planning. It provides a discussion of its hierarchical design and 45 indicators. Using this indicator set, local authorities can assess and enhance UFT sustainability. 