We present a model comparison study that combines multiple integrated assessment models with a reduced-form global air quality model to assess the potential co-benefits of global climate mitigation policies in relation to the World Health Organization (WHO) goals on air quality and health. We include in our assessment, a range of alternative assumptions on the implementation of current and planned pollution control policies. The resulting air pollution emission ranges significantly extend those in the Representative Concentration Pathways. Climate mitigation policies complement current efforts on air pollution control through technology and fuel transformations in the energy system. A combination of stringent policies on air pollution control and climate change mitigation results in 40% of the global population exposed to PM levels below the WHO air quality guideline; with the largest improvements estimated for India, China, and Middle East. Our results stress the importance of integrated multisector policy approaches to achieve the Sustainable Development Goals.