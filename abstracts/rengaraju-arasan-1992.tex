 Forecasting of air travel demand using demand models enables better planning for infrastructure development for air transportation. Modeling for air travel demand is the process of relating a vector of socioeconomic variables and a vector of transport system variables to the demand for air travel. In microanalysis of air travel demand, when models are stratified by origin and destination, the resulting models are called city‐pair models. In this study, a city‐pair model for the demand for domestic air travel has been developed based on the air travel in 40 city pairs in India. The model was calibrated with aggregate cross‐sectional data using multiple linear regression analysis. The model has been validated by two different procedures: the cross‐validation technique and a backward prediction method to know the statistical and logical validity of the model for use in travel demand forecasting. As an illustration of the usefulness of the model in travel demand forecasting, the model was used to predict future demand for air travel. 