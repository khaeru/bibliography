This paper presents the overview of the Shared Socioeconomic Pathways (SSPs) and their energy, land use, and emissions implications. The \{SSPs\} are part of a new scenario framework, established by the climate change research community in order to facilitate the integrated analysis of future climate impacts, vulnerabilities, adaptation, and mitigation. The pathways were developed over the last years as a joint community effort and describe plausible major global developments that together would lead in the future to different challenges for mitigation and adaptation to climate change. The \{SSPs\} are based on five narratives describing alternative socio-economic developments, including sustainable development, regional rivalry, inequality, fossil-fueled development, and middle-of-the-road development. The long-term demographic and economic projections of the \{SSPs\} depict a wide uncertainty range consistent with the scenario literature. A multi-model approach was used for the elaboration of the energy, land-use and the emissions trajectories of SSP-based scenarios. The baseline scenarios lead to global energy consumption of 400–1200 \{EJ\} in 2100, and feature vastly different land-use dynamics, ranging from a possible reduction in cropland area up to a massive expansion by more than 700 million hectares by 2100. The associated annual \{CO2\} emissions of the baseline scenarios range from about 25 GtCO2 to more than 120 GtCO2 per year by 2100. With respect to mitigation, we find that associated costs strongly depend on three factors: (1) the policy assumptions, (2) the socio-economic narrative, and (3) the stringency of the target. The carbon price for reaching the target of 2.6 W/m2 that is consistent with a temperature change limit of 2 °C, differs in our analysis thus by about a factor of three across the \{SSP\} marker scenarios. Moreover, many models could not reach this target from the \{SSPs\} with high mitigation challenges. While the \{SSPs\} were designed to represent different mitigation and adaptation challenges, the resulting narratives and quantifications span a wide range of different futures broadly representative of the current literature. This allows their subsequent use and development in new assessments and research projects. Critical next steps for the community scenario process will, among others, involve regional and sectoral extensions, further elaboration of the adaptation and impacts dimension, as well as employing the \{SSP\} scenarios with the new generation of earth system models as part of the 6th climate model intercomparison project (CMIP6).