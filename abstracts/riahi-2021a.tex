Global emissions scenarios play a critical role in the assessment of strategies to mitigate climate change. The current scenarios, however, are criticized because they feature strategies with pronounced overshoot of the global temperature goal, requiring a long-term repair phase to draw temperatures down again through net negative emissions. Some impacts might not be reversible. Hence, we explore a new set of net-zero CO2 emissions scenarios with limited overshoot. We show that upfront investments are needed in near term for limiting temperature overshoot, but that these would bring long-term economic gains. Our study further identifies alternative configurations of net-zero CO2 emissions systems and the roles of different sectors and regions for balancing sources and sinks. Even without net-negative emissions, carbon dioxide and removal (CDR) is important for accelerating near-term reductions and for providing an anthropogenic sink that can offset the residual emissions in sectors that are hard to abate.
