The problem of endowing large applied general equilibrium models with numerical values for parameters is formidable. For example, a complete set of own- and cross-price elasticities of demands for the MONASH model of the Australianeconomy involves in excess of 50000 items. Invoking the minimal assumptions that demand is generated by utility maximization reduces the load to about 26000 - obviously still a number much too large for unrestrained econometric estimation. To obtain demand systems estimates for a dozen or so generic commodities at a top level of aggregation (categories like ‘food’, ‘clothing and footwear’, etc.), typically Johansen's lead has been followed, and directly additive preferences imposed upon the underlying utility function. With the move beyond one-step linearized solutions of applied general equilibrium models, the functional form of the demand system adopted becomes an issue. The most celebrated of the additive-preference demand systems, the linear expenditure system (LES), has one drawback for empirical work; namely, the constancy of marginal budget shares (MBSs) - a liability shared with the Rotterdam system. To get around this, Theil and Clements used Holbrook Working's Engel specification in conjunction with additive preferences; unfortunately both Working's formulation and Deaton and Muellbauer's AIDS have the problem that, under large changes in real incomes, budget shares can stray outside the [0, 1] interval. It was such behaviour that led Cooper and McLaren to devise systems with better regularity properties. These systems, however, are not globally compatible with any additive preference system. In this paper we specify, and estimate, at the six-commodity level, an implicitly directly additive-preference demand system which allows MBSs to vary as a function of total real expenditure and which is globally regular throughout that part of the the price-expenditure space in which the consumer is at least affluent enough to meet subsistence requirements.
