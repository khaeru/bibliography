This chapter reviews the different pathways which cities are following to become more accessible. By identifying the close link between transport and urban form based on global evidence, it highlights the direct and indirect costs of choices made. It then presents the tipping points which can allow to proceed from sprawling urban development and conventional motorised transport to more compact cities characterised by innovative mobility choices shaped around shared and public transport. The examples used are based on cities worldwide to illustrate emerging trends from both developed and developing countries. Therefore, the recommendations are valuable for a range of stakeholders including local and national policy makers, academics and vehicle manufacturers.