The distribution of wealth in the United States and countries around the world is highly skewed. How does visible economic inequality affect well-off individuals’ support for redistribution? Using a placebo-controlled field experiment, I randomize the presence of poverty-stricken people in public spaces frequented by the affluent. Passersby were asked to sign a petition calling for greater redistribution through a “millionaire’s tax.” Results from 2,591 solicitations show that in a real-world-setting exposure to inequality decreases affluent individuals’ willingness to redistribute. The finding that exposure to inequality begets inequality has fundamental implications for policymakers and informs our understanding of the effects of poverty, inequality, and economic segregation. Confederate race and socioeconomic status, both of which were randomized, are shown to interact such that treatment effects vary according to the race, as well as gender, of the subject.