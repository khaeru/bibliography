In order for the world to stay within the safety threshold of a 2°C increase in average temperature agreed by virtually all governments, the transport sector needs to be decarbonized. The two main obstacles that have prevented this from happening have been the absence of a global legally binding deal and the high relative cost of clean vehicle/energy technologies. The Paris Agreement, which commits countries to reductions of GHG emissions, has virtually solved the first problem and paved the way for countries to implement environmental taxes and subsidies in order to change the relative costs of clean alternatives, which would solve the second problem. These policy actions combined with investment in clean infrastructure and regulation can decarbonize the transport sector.