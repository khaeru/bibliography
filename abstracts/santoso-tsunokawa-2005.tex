This article examines the spatial transferability of mode choice models in developing countries. An evaluation of the updating procedure and sample size are also included in the study. Because of the insufficiency of model coefficients in explaining differences in unmeasured modal attributes, naïvely transferring a model is not recommended. An understanding of the transport characteristics in both the estimation context and the application context is required, in order to justify whether a variable is transferable or not. Four updating procedures – updating alternative specific constants (ASCs), updating ASCs and scale parameter, the combined transfer estimator and Bayesian updating associated with three sets of small sample sizes – are applied to improve transferability. In general, the first three approaches produce significant improvements. It is also proposed that a minimum small sample size of 400 observations is necessary for updating purposes.
