The Government of India has recently announced the National Electric Mobility Mission Plan, which sets ambitious targets for electric vehicle deployment in India. One important barrier to substantial market penetration of EVs in India is the impact that large numbers of EVs will have on an already strained electricity grid. Properly predicting the impact of EVs on the Indian grid will allow better planning of new generation and distribution infrastructure as the EV mission is rolled out. Properly predicting the grid impacts from EVs requires information about the electrical energy consumption of different types of EVs in Indian driving conditions. This study uses detailed vehicle powertrain models to estimate per kilometer electrical consumption for electric scooters, 3-wheelers and different types of 4-wheelers in India. The powertrain modeling methodology is validated against experimental measurements of electrical consumption for a Nissan Leaf. The model is then used to predict electrical consumption for several types of vehicles in different driving conditions. The results show that in city driving conditions, the average electrical consumption is: 33Wh/km for the scooter, 61Wh/km for the 3-wheeler, 84Wh/km for the low power 4-wheeler, and 123Wh/km for the high power 4-wheeler. For highway driving conditions, the average electrical consumption is: 133Wh/km for the low power 4-wheeler, and 165Wh/km for the high power 4-wheeler. The impact of variations in several parameters are modeled, including the impact of different driving conditions, different levels of loading by air conditions and other ancillary components, different total vehicle masses, and different levels of motor operating efficiency.