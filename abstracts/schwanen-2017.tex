Geographical scholarship on transport has been boosted by the emergence of big data and advances in the analysis of complex networks in other disciplines, but these developments are a mixed blessing. They allow transport as object of analysis to exist in new ways and raise the profile of geography in interdisciplinary spaces dominated by physics and complexity science. Yet, they have also brought back concerns over the privileging of generality over particularity. This is because they have once more made acceptable and even normalized a focus on supposedly universal laws that explain the functioning of mobility systems and on space and time independent explanations of hierarchies, inequalities and vulnerabilities in transport systems and patterns. Geographical scholarship on transport should remain open to developments in big data and network science but would benefit from more critical reflexivity on the limitations and the historical and geographical situatedness of big data and on the conceptual shortcomings of network science. Big data and network analysis need to be critiqued and re-appropriated, and examples of how this can be done are starting to emerge. Openness, critique and re-appropriation are especially important in a context where transport geography decentralizes away from its Euro-American core, and the development pathways of transport and mobility in localities beyond that core deserve their own, unique explanations.