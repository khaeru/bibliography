This article addresses a number of key problems commonly confronted in the literature on international demand analysis. These include data issues and requirements, multistage budgeting, outliers, group heteroskedasticity, and model selection. A two‐stage demand system is fit to International Comparison Programme data for 114 countries for nine aggregate categories and eight food sub‐categories of goods. Outliers are identified and omitted from the sample. Parameter estimates for the two stages are obtained with a maximum‐likelihood procedure that corrects for group heteroskedasticity. Country‐specific income and own‐price elasticities are calculated and indicate that poor countries are more responsive to changes in income and prices than rich countries. We also find evidence for the strong version of Engel's law; when income doubles, the budget share of food declines by approximately 0.10.