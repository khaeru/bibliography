Summary Lifestyles are changing due to information technology and other socio-technological trends. We study the energy effects induced by lifestyle shifts via tradeoffs in time spent in performing activities. We use the American Time Use Survey to find changes in times performing different activities from 2003 to 2012. The results show that Americans are spending considerably more time at home (7.8 days more in 2012 compared with 2003). This increased home time is counterbalanced by decreased time spent traveling (1.2 days less in 2012 versus 2003) and in non-residential buildings (6.7 days less in 2012 versus 2003). Increased residential time is mainly due to increased work at home, video watching, and computer use. Decomposition analysis is then used to estimate effects on energy consumption, indicating that more time at home and less on travel and in non-residential buildings reduced national energy demand by 1,700 trillion \{BTU\} in 2012, 1.8% of the national total.