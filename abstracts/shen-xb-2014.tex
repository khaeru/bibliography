As stricter standards for diesel vehicles are implemented in China, and the use of diesel trucks is forbidden in urban areas, determining the contribution of light-duty gasoline vehicles (LDGVs) to on-road PM2.5 emissions in cities is important. Additionally, in terms of particle number and size, particulates emitted from \{LDGVs\} have a greater health impact than particulates emitted from diesel vehicles. In this work, we measured PM2.5 emissions from 20 \{LDGVs\} in Beijing, using an improved combined on-board emission measurement system. We compared these measurements with those reported in previous studies, and estimated the contribution of \{LDGVs\} to on-road PM2.5 emissions in Beijing. The results show that the PM2.5 emission factors for LDGVs, complying with European Emission Standards Euro-0 through Euro-4 were: 117.4 ± 142, 24.1 ± 20.4, 4.85 ± 7.86, 0.99 ± 1.32, 0.17 ± 0.15 mg/km, respectively. Our results show a significant decline in emissions with improving vehicle technology. However, this trend is not reflected in recent emission inventory studies. The daytime contributions of \{LDGVs\} to PM2.5 emissions on highways, arterials, residential roads, and within urban areas of Beijing were 44\%, 62\%, 57\%, and 57\%, respectively. The contribution of \{LDGVs\} to PM2.5 emissions varied both for different road types and for different times.