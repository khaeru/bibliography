A database of real-world diesel vehicle emission factors, based on type and technology, has been developed following tests on more than 300 diesel vehicles in China using a portable emission measurement system. The database provides better understanding of diesel vehicle emissions under actual driving conditions. We found that although new regulations have reduced real-world emission levels of diesel trucks and buses significantly for most pollutants in China, NO$_X$ emissions have been inadequately controlled by the current standards, especially for diesel buses, because of bad driving conditions in the real world. We also compared the emission factors in the database with those calculated by emission factor models and used in inventory studies. The emission factors derived from COPERT (Computer Programmer to calculate Emissions from Road Transport) and MOBILE may both underestimate real emission factors, whereas the updated COPERT and PART5 (Highway Vehicle Particulate Emission Modeling Software) models may overestimate emission factors in China. Real-world measurement results and emission factors used in recent emission inventory studies are inconsistent, which has led to inaccurate estimates of emissions from diesel trucks and buses over recent years. This suggests that emission factors derived from European or US-based models will not truly represent real-world emissions in China. Therefore, it is useful and necessary to conduct systematic real-world measurements of vehicle emissions in China in order to obtain the optimum inputs for emission inventory models.
