Railway is one of the most efficient and environmental-friendly ways to transport people and goods. High-speed railway has been developing rapidly and the railway mileage has increased by 21.18% in China during the period of 2006–2011 and thus it is interesting to evaluate whether the railway transportation has reduced the environmental impact of transport in China. In this paper, we first use a non-radial DEA under managerial disposability to measure the environmental efficiency of 30 regions in China; then we propose a panel beta regression with fixed effects to model the impact of railway transportation on environmental efficiency. The results indicate that the environmental efficiency slowly increased during 2006–2011 and it exhibits regional disparities with the eastern area having the highest environmental efficiency and the western area being the lowest one; Moreover, we also find a significant positive impact of railway transportation on higher environmental efficiency.