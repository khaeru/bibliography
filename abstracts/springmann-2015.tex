China's Twelfth Five-Year Plan (2011--2015) aims to achieve a national carbon intensity reduction of 17 {\%} through differentiated targets at the provincial level. Allocating the national target among China's provinces is complicated by the fact that more than half of China's national carbon emissions are embodied in interprovincial trade, with the relatively developed eastern provinces relying on the center and west for energy-intensive imports. This study develops a consistent methodology to adjust regional emissions-intensity targets for trade-related emissions transfers and assesses its economic effects on China's provinces using a regional computable-general-equilibrium (CGE) model of the Chinese economy. This study finds that in 2007 China's eastern provinces outsource 14 {\%} of their territorial emissions to the central and western provinces. Adjusting the provincial targets for those emissions transfers increases the reduction burden for the eastern provinces by 60 {\%}, while alleviating the burden for the central and western provinces by 50 {\%} each. The CGE analysis indicates that this adjustment could double China's national welfare loss compared to the homogenous and politics-based distribution of reduction targets. A shared-responsibility approach that balances production-based and consumption-based emissions responsibilities is found to alleviate those unbalancing effects and lead to a more equal distribution of economic burden among China's provinces.