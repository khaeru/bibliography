This paper presents a critical history of the environmental Kuznets curve (EKC). The EKC proposes that indicators of environmental degradation first rise, and then fall with increasing income per capita. Recent evidence shows however, that developing countries are addressing environmental issues, sometimes adopting developed country standards with a short time lag and sometimes performing better than some wealthy countries, and that the EKC results have a very flimsy statistical foundation. A new generation of decomposition and efficient frontier models can help disentangle the true relations between development and the environment and may lead to the demise of the classic EKC.
