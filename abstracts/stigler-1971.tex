The potential uses of public resources and powers to improve the economic status of economic groups (such as industries and occupations) are analyzed to provide a scheme of the demand for regulation. The characteristics of the political process which allow relatively small groups to obtain such regulation is then sketched to provide elements of a theory of supply of regulation. A variety of empirical evidence and illustration is also presented.
