It is commonly agreed that the level of income and prices are crucial determinants of the consumption of motor gasoline. The respective long run price and income elasticities are regularly calculated using cross sectional models. Despite the acknowledgement of the role of income distribution, it plays no role in intercountry cross sectional models. This is due to a lack of appropriate data. This paper shows that the omission of distributional characteristics provides misleading elasticities. Using available distributional measures this paper is referring to an income threshold, which is crucial to the acquisition of an automobile. It is shown that on the one hand, in poor countries an unequal income distribution is needed to enable at least some people to buy automobiles. On the other hand, in wealthy countries an unequal income distribution would exclude some people from acquiring automobiles. Hence, depending on the income level, inequality has a diverging impact on the ability to buy durable goods. The second part of this paper develops a pooled 90-country model to examine this approach empirically. It could be shown that distribution variables are highly significant to explain the demand for automobiles and motor gasoline. Moreover, the consideration of the distribution of income leads to a considerable decrease in income elasticity values. This is mainly due to the positive correlation between income level and income equality within the sample.
