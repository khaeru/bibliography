Using data from 161 Chinese cities, this paper investigates the effects of various dimensions of urban spatial structure on the ownership and commute mode split of automobile. Results confirm the positive effects of city size on auto ownership and mode split and the negative effect of density on auto ownership. Echoing a small number of studies, this research discovers the seemingly counterintuitive effect of jobs-housing balance on the use of automobiles, probably due to the potential advantage of public transit relative to driving in dense and congested Chinese cities. Cities should emphasize public transit and maintain density in the future.
