Urbanization, one of the most obvious characteristics of economic growth in China, has an apparent “lock-in effect” on residential energy consumption pattern. It is expected that residential sector would become a major force that drives China's energy consumption after urbanization process. We estimate price and expenditure elasticities of residential energy demand using data from China's Residential Energy Consumption Survey (CRECS) that covers households at different income levels and from different regional and social groups. Empirical results from the Almost Ideal Demand System model are in accordance with the basic expectations: the demands for electricity, natural gas and transport fuels are inelastic in the residential sector due to the unreasonable pricing mechanism. We further investigate the sensitivities of different income groups to prices of the three types of energy. Policy simulations indicate that rationalizing energy pricing mechanism is an important guarantee for energy sustainable development during urbanization. Finally, we put forward suggestions on energy pricing reform in the residential sector based on characteristics of China's undergoing urbanization process and the current energy consumption situations.