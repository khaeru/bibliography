The complex large-scale integrated open systems (CLIOS) process is an overarching mechanism for considering systems, especially those exhibiting nested complexity, in which both the physical and the institutional aspects are complex. A special case of the CLIOS process is regional strategic transportation planning. New technologies, such as intelligent transportation systems, allow consideration of the planning, management, operations, and maintenance of transportation systems at the regional scale. Although the technological issues in advancing to this scale have proved tractable, the institutional issues concerned with deploying these systems across political jurisdictions with different measures of performance and different cultural perspectives has proved quite difficult. This paper explores processes for studying these institutional questions and integrating a number of concepts into the process: technological change, sustainability, real options, supply chain management, the various transport modes, and social, political, and economic factors. The paper serves as a case study of tailoring the CLIOS process into a process specifically designed to systematically address particular issues, in this case, regional strategic transportation planning.
