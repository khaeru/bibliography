This paper studies the distributional effects of rising energy costs for households. In contrast to most of the previous literature, our specification differentiates between electricity and heating demand and still models other consumption goods in realistic detail. We use a yet unexplored data-set on household expenditures in Germany and extend the recently developed EASI demand system for the analysis of inequality and welfare at the individual and social level. The EASI system reveals non-linearity of Engel curves which – when neglected – can lead to biased estimates of distributional effects. We find that increases in heating prices are more regressive than those in electricity. Furthermore, current proposals for social tariffs are found to be less effective than targeted compensation schemes.