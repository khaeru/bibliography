Common approaches to assess the evolution of \{CO2\} emissions from road vehicles are usually based on (a) estimates of future fleet composition, where most approaches consider vehicles at a rather aggregated level, and (b) emission factors, which are either based on \{CO2\} certification data or statistically-provided functional relationships obtained from real world test data, or a combination of the two. This approach has certain limitations in capturing the effect of new technologies on \{CO2\} emission related policy initiatives. The present study proposes a new method for the detailed calculation of the European light duty vehicle fleet \{CO2\} emissions, which could help to overcome such limitations, achieve better results when making \{CO2\} emissions projections and better support future policies. Simulation at individual vehicle level is combined with fleet composition data, retrieved from the official European \{CO2\} emissions monitoring database, and publicly available data regarding individual vehicle characteristics in order to calculate vehicle \{CO2\} emissions and fuel consumption over different conditions and vehicle configurations. The methodology is applied to analyze and assess the impact of the introduction of the new certification procedure, the Worldwide Light duty vehicle Test Procedure (WLTP), on the European car fleet \{CO2\} emissions. Results show an average \{WLTP\} to \{NEDC\} \{CO2\} emissions ratio of approximately 1.2. The increases in \{CO2\} emissions are higher for cars exhibiting lower \{NEDC\} emission values (additional 29 and 25 gCO2/km for vehicles emitting 100 and 119 gCO2/km, respectively). At higher emission levels (about 250 \{CO2\} g/km) \{WLTP\} and \{NEDC\} present comparable results. Three possible scenarios for the translation of projected \{NEDC\} \{CO2\} emissions to WLTP-based ones are quantified.