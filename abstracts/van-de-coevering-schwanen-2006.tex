The work by Newman, Kenworthy and colleagues on the link between land use, transportation systems and travel patterns and energy use has been received enthusiastically but also criticised strongly. In this paper concerns are expressed about the role accorded to individual travellers and the wider space-time context of cities in the empirical-analytical work by Kenworthy and colleagues. To investigate the seriousness of these concerns, the data collected by Kenworthy and colleagues for European, Canadian and US cities in 1990 have been augmented with information on housing, urban development history and the sociodemographic situation. Regression models are described in which the role of urban form is investigated while account is taken of other relevant factors. The empirical analysis suggests that the space-time context of cities should be taken into account in aggregate-level comparisons of the relations between urban form and transport. Policy recommendations based on the original data may be reconsidered and tailored to the space-time context and population characteristics of cities.