We estimated global fine particulate matter (PM2.5) concentrations using information from satellite-, simulation- and monitor-based sources by applying a Geographically Weighted Regression (GWR) to global geophysically based satellite-derived PM2.5 estimates. Aerosol optical depth from multiple satellite products (MISR, MODIS Dark Target, MODIS and SeaWiFS Deep Blue, and MODIS MAIAC) was combined with simulation (GEOS-Chem) based upon their relative uncertainties as determined using ground-based sun photometer (AERONET) observations for 1998–2014. The GWR predictors included simulated aerosol composition and land use information. The resultant PM2.5 estimates were highly consistent (R2 = 0.81) with out-of-sample cross-validated PM2.5 concentrations from monitors. The global population-weighted annual average PM2.5 concentrations were 3-fold higher than the 10 μg/m3 WHO guideline, driven by exposures in Asian and African regions. Estimates in regions with high contributions from mineral dust were associated with higher uncertainty, resulting from both sparse ground-based monitoring, and challenging conditions for retrieval and simulation. This approach demonstrates that the addition of even sparse ground-based measurements to more globally continuous PM2.5 data sources can yield valuable improvements to PM2.5 characterization on a global scale.