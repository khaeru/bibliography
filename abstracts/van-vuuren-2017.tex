This paper describes the possible developments in global energy use and production, land use, emissions and climate changes following the SSP1 storyline, a development consistent with the green growth (or sustainable development) paradigm (a more inclusive development respecting environmental boundaries). The results are based on the implementation using the IMAGE 3.0 integrated assessment model and are compared with a) other IMAGE implementations of the SSPs (SSP2 and SSP3) and b) the SSP1 implementation of other integrated assessment models. The results show that a combination of resource efficiency, preferences for sustainable production methods and investment in human development could lead to a strong transition towards a more renewable energy supply, less land use and lower anthropogenic greenhouse gas emissions in 2100 than in 2010, even in the absence of explicit climate policies. At the same time, climate policy would still be needed to reduce emissions further, in order to reduce the projected increase of global mean temperature from 3°C (SSP1 reference scenario) to 2 or 1.5°C (in line with current policy targets). The SSP1 storyline could be a basis for further discussions on how climate policy can be combined with achieving other societal goals.