We evaluate the pollution and labor supply reductions from Beijing's driving restrictions. Causal effects are identified from both time-series and spatial variation in air quality and intra-day variation in television viewership. Based on daily data from multiple monitoring stations, air pollution falls 21% during one-day-per-week restrictions. Based on hourly television viewership data, viewership during the restrictions increases by 9 to 17% for workers with discretionary work time but is unaffected for workers without, consistent with the restrictions' higher per-day commute costs reducing daily labor supply. We provide possible reasons for the policy's success, including evidence of high compliance based on parking garage entrance records.