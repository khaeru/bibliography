This article contributes to the controversial debate over the effect of spatial organization on CO2 emissions by investigating the potential of infrastructure measures that favour lower mobility in achieving the transition to a low-carbon economy. The energy–economy–environment (E3) IMACLIM-R model is used to provide a detailed representation of passenger and freight transportation. Unlike many of the E3 models used to simulate mitigation options, IMACLIM-R represents both the technological and behavioural determinants of mobility. By comparing business-as-usual, carbon price only, and carbon price combined with transport policy scenarios, it is demonstrated that the measures that foster a modal shift towards low-carbon modes and a decoupling of mobility needs from economic activity significantly modify the sectoral distribution of mitigation efforts and reduce the level of carbon tax necessary to reach a given climate target relative to a ‘carbon price only’ policy. Policy relevance Curbing carbon emissions from transport activities is necessary in order to reach mitigation targets, but it poses a challenge for policy makers. The transport sector has two peculiarities: a weak ability to react to standard pricing measures (which encourages richer policy interventions) and a dependence on long-lived infrastructure (which imposes a delay between policy interventions and effective action). To address these problems, a framework is proposed for analysing the role of transport-specific measures adopted complementarily to carbon pricing in the context of international climate policies. Consideration is given to alternative approaches such as infrastructure measures designed to control mobility through less mobility-intensive denser agglomerations, investment reorientation towards public mode, and logistics reorganization towards less mobility-dependent production processes. Such measures can significantly reduce transport emissions in the long term and hence would moderate an increase in the carbon price and reduce its more important detrimental impacts on the economy.