China is currently experiencing accelerated urbanization, with total urban population accounting for 52.6\% of the country's total population in 2012. The next 10 years is foreseen to be characterized by the rural-to-urban migration of more than 300 million citizens. The consequent pressure on the environment compels small and medium-sized cities to accommodate the influx of migrants—a situation that inevitably brings new challenges to public utility management in the country. Most of these cities lack systematic management and consistent standards in the formulation of public transportation policy because of a vague decision-making mechanism. We empirically investigate the decision-making process for public transportation policy in China's medium-sized cities, focusing specifically on the ownership reform of the public transportation system in Huizhou, Guangdong. We apply Kingdon's multi-stream model and extensively interview stakeholders who shape public transit policy in the study area. On these bases, we discuss how the three streams—problems, policy, and politics—converge and initiate the reform of public transportation systems. Kingdon's model enables the identification of weak links in the transportation management systems of China's medium-sized cities.
