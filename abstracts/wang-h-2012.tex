Transport sector accounts for about 8% of total energy consumption in China and this share will likely increase in the visible future. Improving energy efficiency has been considered as a major way for reducing transport energy use, whereas its effectiveness might be affected by the rebound effect. This paper estimates the direct rebound effect for passenger transport in urban China by using the linear approximation of the Almost Ideal Demand System model and simulation analysis. Our empirical results reveal the existence of direct rebound effect for passenger transport in urban China. A majority of the expected reduction in transport energy consumption from efficiency improvement could be offset due to the existence of rebound effect. We have further investigated the relationship between the magnitude of direct rebound effect and households' expenditure. It was found that the direct rebound effect for passenger transport tends to decline with the increase of per capita household consumption expenditure.