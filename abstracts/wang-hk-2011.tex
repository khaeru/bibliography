In this paper, CO$_2$ and pollutant emissions of PCs in China from 2000 to 2005 were calculated based on a literature review and measured data. The future trends of PC emissions were also projected under three scenarios to explore the reduction potential of possible policy measures. Estimated baseline emissions of CO, HC, NO$_X$, PM$_{10}$ and CO$_2$ were respectively 3.16×106, 5.14×105, 3.56×105, 0.83×104 and 9.14×107\&\#xa0;tons for China’s PCs in 2005 with an uneven distribution among provinces. Under a no improvement (NI) scenario, PC emissions of CO, HC, NO$_X$, PM$_{10}$ and CO$_2$ in 2020 are respectively estimated to be 4.5, 2.5, 2.5, 7.9 and 8.0 times that of 2005. However, emissions other than CO$_2$ from PCs are estimated to decrease nearly 70\% by 2020 compared to NI scenario mainly due to technological improvement linked to the vehicle emissions standards under a recent policy (RP) scenario. Fuel economy (FE) enhancement and the penetration of advanced propulsion/fuel systems could be co-benefit measures to control CO$_2$ and pollutant emissions for the mid and long terms. Significant variations were found in PC emission inventories between different studies primarily due to uncertainties in activity levels and/or emission factors (EF).
