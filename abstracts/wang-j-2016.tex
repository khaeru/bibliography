China’s air transportation has experienced rapid growth and major reforms in the past three decades, some of which have been partially successful and are still ongoing today. The paper aims to analyze China’s air deregulation experience over the last two decades and its impact on airline competition from a geographical perspective. After the establishment of the “Big Three” in 2002, the paper reveals that there has been a trade-off between the extent of deregulation and airline competition in China because the central government has tended to strengthen the “Big Three” rather than totally open the market to private and locally owned airlines. The paper uses each airline group as the basic unit of analysis and reveals that (1) the air market has been more concentrated in the “Big Three” as a result of the process of air deregulation; (2) airline competition in over two thirds of the airports and one half of the routes has increased in the last 18years, but the core airports and trunk routes are chiefly dominated by the “Big Three”. The peripheral airports and thin routes have been operated by private and locally owned airlines; and (3) regionally, airline competition has occurred in most airports of the eastern region, and it is more intense than in the central and western regions. But even here, competition in the eastern region has however decreased in 1994–2012. The three main contributions of the paper are: (1) the use of two measures of competition in the airline market; (2) the analysis of the historical evolution of competition; and (3) an understanding the role of the geography of competition in the Chinese airline market.