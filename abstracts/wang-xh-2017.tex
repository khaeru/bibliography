Rural household energy consumption is an important component of national energy consumption and plays an important role in rural social and ecological environment developments. In this paper, energy consumption of 1440 households in 8 typical counties of 8 China's economic zones was investigated. The investigation data analysis revealed significant difference of different economic zones in rural household energy consumption level and structure. For 8 studying counties, the annual average energy consumption per capita was 26.7GJ, 10.4GJ the lowest (Shanghang County) and 86.6GJ the highest (Shulan County). In energy consumption, straw, biogas, fuel wood and electricity accounted for 44.33%, 23.13%, 12.79% and 9.61%, respectively. Rural families with high incomes preferred commercial energies (e.g. electricity and liquefied petroleum gas (LPG)) to biomass energy (e.g. straw and fuel wood). The traditional biomass energy is still the main energy source for China's rural household. Research results provide references to understand current situations and future development of China's rural household energy consumption, and formulate related energy and environmental policies.