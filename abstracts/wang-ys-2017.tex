China's plug-in electric vehicle (PEV) sales, comprising both battery electric and plug-in hybrid vehicles, surged 343% in 2015, and are expected to reach 2 million by 2020. Two factors are crucial to this sudden transformation: 1) massive central and local government subsidies, and 2) huge non-monetary incentives via exemptions from restrictions on vehicle ownership in Beijing, Shanghai, and elsewhere. Innovative business models and greatly expanded vehicle offerings, especially by local Chinese manufacturers, also helped accelerate PEV sales and infrastructure deployment. However, continued sales growth is threatened by persistent regional protectionism, the unsustainability of these large subsidies, and widely reported cheating by some automakers. We suggest some innovative policies that China might pioneer and transfer elsewhere.