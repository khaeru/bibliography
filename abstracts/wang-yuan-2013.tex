With the rapid motorization in China, parking has become increasingly difficult and costly for automobile users. However, the effects of parking on the society go far beyond vehicle owners' costs. To inform decision makers in China and cities in similar motorizing societies, this study describes the market and policy trends of automobile parking in Chinese cities. Available data show that the gap between supply and demand in parking has enlarged, while most city governments have little experience and are institutionally unprepared for the proper planning, regulation, and management of parking. International experience and the Chinese problems call for a reform in urban parking management in order to promote sustainable urban transportation and maximize social welfare. This paper offers policy and planning suggestions regarding on- and off-street parking.
