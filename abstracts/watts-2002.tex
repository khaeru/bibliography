The origin of large but rare cascades that are triggered by small initial shocks is a phenomenon that manifests itself as diversely as cultural fads, collective action, the diffusion of norms and innovations, and cascading failures in infrastructure and organizational networks. This paper presents a possible explanation of this phenomenon in terms of a sparse, random network of interacting agents whose decisions are determined by the actions of their neighbors according to a simple threshold rule. Two regimes are identified in which the network is susceptible to very large cascades—herein called global cascades—that occur very rarely. When cascade propagation is limited by the connectivity of the network, a power law distribution of cascade sizes is observed, analogous to the cluster size distribution in standard percolation theory and avalanches in self-organized criticality. But when the network is highly connected, cascade propagation is limited instead by the local stability of the nodes themselves, and the size distribution of cascades is bimodal, implying a more extreme kind of instability that is correspondingly harder to anticipate. In the first regime, where the distribution of network neighbors is highly skewed, it is found that the most connected nodes are far more likely than average nodes to trigger cascades, but not in the second regime. Finally, it is shown that heterogeneity plays an ambiguous role in determining a system's stability: increasingly heterogeneous thresholds make the system more vulnerable to global cascades; but an increasingly heterogeneous degree distribution makes it less vulnerable.
