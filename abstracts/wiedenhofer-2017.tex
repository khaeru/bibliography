Households/' carbon footprints are unequally distributed among the rich and poor due to differences in the scale and patterns of consumption. We present distributional focused carbon footprints for Chinese households and use a carbon-footprint-Gini coefficient to quantify inequalities. We find that in 2012 the urban very rich, comprising 5\% of population, induced 19\% of the total carbon footprint from household consumption in China, with 6.4 tCO2/cap. The average Chinese household footprint remains comparatively low (1.7 tCO2/cap), while those of the rural population and urban poor, comprising 58\% of population, are 0.5-1.6 tCO2/cap. Between 2007 and 2012 the total footprint from households increased by 19\%, with 75\% of the increase due to growing consumption of the urban middle class and the rich. This suggests that a transformation of Chinese lifestyles away from the current trajectory of carbon-intensive consumption patterns requires policy interventions to improve living standards and encourage sustainable consumption.
