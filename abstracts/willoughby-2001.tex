Because of the rapid economic growth it sustained over the last 40 years and the small physical space at its disposal, Singapore has had to give special attention to managing the process of motorization—the spread of private motor vehicle ownership and use. Despite the inevitable imperfections of the policies adopted—and, more seriously, of related land-use and resettlement policies—the motorization restraints had no major negative side-effect on economic growth and generated substantial funds for the improvement of social welfare. The package of policies applied merits close examination by developing- and transition-country cities that need urgently to find new ways of raising financial resources to meet the huge needs arising from population growth and resettlement.
