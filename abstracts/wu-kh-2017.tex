Congestion charging is being considered as a potential measure to address the issue of substantially increased traffic congestion and vehicle emissions in Beijing. This study assessed the impact of congestion charging on traffic and emissions in Beijing using macroscopic traffic simulation and vehicle emissions calculation. Multiple testing scenarios were developed with assumptions in different charging zone sizes, public transit service levels and charging methods. Our analysis results showed that congestion charging in Beijing may increase public transit use by approximately 13%, potentially reduce CO and HC emissions by 60–70%, and reduce NOx emissions by 35–45% within the charging zone. However, congestion charging may also result in increased travel activities and emissions outside of the charging zone and a slight increase in emissions for the entire urban area. The size of charging zone, charging method, and charging rate are key factors that directly influence the impact of congestion charging; improved public transit service needs to be considered as a complementary approach with congestion charging. This study is used by Beijing Transportation Environment and Energy Center (BTEC) as reference to support the development of Beijing’s congestion charging policy and regulation.