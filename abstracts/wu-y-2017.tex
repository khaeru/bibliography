The large (26-fold over the past 25 years) increase in the on-road vehicle fleet in China has raised sustainability concerns regarding air pollution prevention, energy conservation, and climate change mitigation. China has established integrated emission control policies and measures since the 1990s, including implementation of emission standards for new vehicles, inspection and maintenance programs for in-use vehicles, improvement in fuel quality, promotion of sustainable transportation and alternative fuel vehicles, and traffic management programs. As a result, emissions of major air pollutants from on-road vehicles in China have peaked and are now declining despite increasing vehicle population. As might be expected, progress in addressing vehicle emissions has not always been smooth and challenges such as the lack of low sulfur fuels, frauds over production conformity and in-use inspection tests, and unreliable retrofit programs have been encountered. Considering the high emission density from vehicles in East China, enhanced vehicle, fuel and transportation strategies will be required to address vehicle emissions in China. We project the total vehicle population in China to reach 400–500 million by 2030. Serious air pollution problems in many cities of China, in particular high ambient PM2.5 concentration, have led to pressure to accelerate the progress on vehicle emission reduction. A notable example is the draft China 6 emission standard released in May 2016, which contains more stringent emission limits than those in the Euro 6 regulations, and adds a real world emission testing protocol and a 48-h evaporation testing procedure including diurnal and hot soak emissions. A scenario (PC[1]) considered in this study suggests that increasingly stringent standards for vehicle emissions could mitigate total vehicle emissions of HC, CO, \{NOX\} and PM2.5 in 2030 by approximately 39\%, 57\%, 59\% and 79\%, respectively, compared with 2013 levels. With additional actions to control the future light-duty passenger vehicle population growth and use, and introduce alternative fuels and new energy vehicles, the China total vehicle emissions of HC, CO, \{NOX\} and PM2.5 in 2030 could be reduced by approximately 57\%, 71\%, 67\% and 84\%, respectively, (the PC[2] scenario) relative to 2013. This paper provides detailed policy roadmaps and technical options related to these future emission reductions for governmental stakeholders.