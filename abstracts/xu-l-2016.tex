This paper proposes a new typology that classifies innovation policy instruments into two dimensions: government-selection versus market-selection, and producer-orientation versus consumer-orientation. Such a typology articulates the importance of consumer behavior in the policy design for a transition, and the relevance for the market to select target subjects of policy during the deployment stage of clean technology innovation. We apply this typology to policy instruments of China's new energy vehicle (NEV) industry between 1991 and 2015 in order to explain the industry's rapid growth. The focus of China's policy mix has transited from government-selection to market-selection, and from producer-orientation to consumer-orientation. Other than the new typology, this paper traces the entire history of policy transition within China's \{NEV\} industry, and finds the transition to be a result of policy learning, thus contributing to future empirical studies of this industry.