To alleviate congestion, many Chinese cities have adopted either one of the two car ownership policies, namely, license plate auction or license plate lottery, to limit the number of cars on the road. In an effort to address the criticism associated with administering a single car ownership policy, cities are considering the possibility of carrying out both policies simultaneously, so that residents can choose whether to pay for the license plate through an auction or get it for free from a lottery but with a longer wait time. We study residents’ preferences toward the two car ownership policies when both are administered at the same time, a problem that has not been investigated in the literature. We then examine the influence of car ownership policies on the choice of electric cars, which is also new to the literature. Using data collected from a stated preference survey, we estimate mixed logit models using the hierarchical Bayes approach based on the Markov Chain Monte Carlo method. Results show that strong preference heterogeneity exists in respondents’ policy choice. We proceed to conduct regression analysis to explain the variations in the preferences toward license plate auction and electric cars. Our main results include: (1) We find that prospective car buyers in Beijing and Shanghai are willing to bid 27,000 yuan and 49,000 yuan to shorten their wait time to get car license plates by one year, respectively; (2) The subsidy to electric cars can be reduced by 102,000 yuan in Beijing and 85,000 yuan in Shanghai if the wait time for an electric car license plate is shortened by one year; (3) Car buyers in favor of license plate auction are those who are from high-income households, who are not buying their first cars, and who are below 30 or above 40 years old; and (4) When promoting the adoption of electric cars, policy incentives, such as making it easier to obtain an electric car license plate and providing attractive subsidies, are as important as the technological advancement electric car manufacturers strive to make, such as improving the driving range of electric cars.