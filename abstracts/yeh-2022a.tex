Transport accounts for 24% of global CO2 emissions from fossil fuels. Governments face challenges in developing feasible and equitable mitigation strategies to reduce energy consumption and manage the transition to low-carbon transport systems. To meet the local and global transport emission reduction targets, policymakers need more realistic/sophisticated future projections of transport demand to better understand the speed and depth of the actions required to mitigate greenhouse gas emissions. In this paper, we argue that the lack of access to high-quality data on the current and historical travel demand and interdisciplinary research hinders transport planning and sustainable transitions toward low-carbon transport futures. We call for a greater interdisciplinary collaboration agenda across open data, data science, behaviour modelling, and policy analysis. These advancemets can reduce some of the major uncertainties and contribute to evidence-based solutions toward improving the sustainability performance of future transport systems. The paper also points to some needed efforts and directions to provide robust insights to policymakers. We provide examples of how these efforts could benefit from the International Transport Energy Modeling Open Data project and open science interdisciplinary collaborations.
