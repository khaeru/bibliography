Rapidly growing energy demand from China's transportation sector in the last two decades have raised concerns over national energy security, local air pollution, and carbon dioxide (CO$_2$) emissions, and there is broad consensus that China's transportation sector will continue to grow in the coming decades. This paper explores the future development of China's transportation sector in terms of service demands, final energy consumption, and CO$_2$ emissions, and their interactions with global climate policy. This study develops a detailed China transportation energy model that is nested in an integrated assessment model—Global Change Assessment Model (GCAM)—to evaluate the long-term energy consumption and CO$_2$ emissions of China's transportation sector from a global perspective. The analysis suggests that, without major policy intervention, future transportation energy consumption and CO$_2$ emissions will continue to rapidly increase and the transportation sector will remain heavily reliant on fossil fuels. Although carbon price policies may significantly reduce the sector's energy consumption and CO$_2$ emissions, the associated changes in service demands and modal split will be modest, particularly in the passenger transport sector. The analysis also suggests that it is more difficult to decarbonize the transportation sector than other sectors of the economy, primarily owing to its heavy reliance on petroleum products.
