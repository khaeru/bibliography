Transport will be the strongest growing energy demand sector in the future, especially in developing countries like China, and it needs more attention. The evolution of transport structure is very important in the dynamic of transport development, and therefore worth emphasis. In this study, a modal split model maximizing spatial welfare and constrained by travel money budget and time budget is developed. This approach differs from the general econometric-based approach used in most existing macro transport studies and deals with the cost and speed of transport modes as important variables explicitly. The model is then applied to China's transport sector together with sensitivity test despite many data problems. The decomposition of energy consumption generated from bottom-up model based on this modal split identified the importance of modal split and turnover expansion in the next 30years, which should be a stronger area of focus in transportation studies.