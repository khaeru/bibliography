China is confronted with the dual task of developing its national economy and protecting its ecological environment. Since the 1980s, China's policies on environmental protection and sustainable development have experienced five changes: (1) progression from the adoption of environmental protection as a basic state policy to the adoption of sustainable development strategy; (2) changing focus from pollution control to ecological conservation equally; (3) shifting from end-of-pipe treatment to source control; (4) moving from point source treatment to regional environmental governance; and (5) a turn away from administrative management-based approaches and towards a legal means and economic instruments-based approach. Since 1992, China has set down sustainable development as a basic national strategy. However, environmental pollution and ecological degradation in China have continued to be serious problems and have inflicted great damage on the economy and quality of life. The beginning of the 21st century is a critical juncture for China's efforts towards sustaining rapid economic development, intensifying environmental protection efforts, and curbing ecological degradation. As the largest developing country, China's policies on environmental protection and sustainable development will be of primary importance not only for China, but also the world. Realizing a completely well-off society by the year 2020 is seen as a crucial task by the Chinese government and an important goal for China's economic development in the new century, however, attaining it would require a four-fold increase over China's year 2000 GDP. Therefore, speeding up economic development is a major mission during the next two decades and doing so will bring great challenges in controlling depletion of natural resources and environmental pollution. By taking a critical look at the development of Chinese environmental policy, we try to determine how best to coordinate the relationship between the environment and the economy in order to improve quality of life and the sustainability of China's resources and environment. Examples of important measures include: adjustment of economic structure, reform of energy policy, development of environmental industry, pollution prevention and ecological conservation, capacity building, and international cooperation and public participation.
