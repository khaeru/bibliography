China has implemented numerous policies to develop electric vehicles (EVs). This paper focuses on evaluating the performance of these policies by systematically collecting and sorting them. These policies can be divided into three categories: finance policy, infrastructure promotion, and research and development (R&D) investment. In addition, the defects of EV policy mechanism are defined by being compared with the policy mechanisms in other countries. The results show that EV policy mechanism should be improved in China. For example, the subsidy and taxation policies are limited, and the goal formulation for EV infrastructure and the input for R&D investment are also unreasonable. Moreover, China need to establish uniform standards for the EV charging infrastructure and a charging pricing mechanism for EVs. Finally, we propose valuable suggestions to improve the performance of EV policies according to the empirical analysis.