This study develops a theoretical framework of green consumer behavior to determine the effects of personal influence, knowledge of green consumption, attitudes toward green consumption, internal and external moderators and examines whether these effects differ significantly among purchasing, using and recycling behaviors. Correlation analysis and multiple regression are applied to assess data collected by a questionnaire survey. The results indicate that attitudes are the most significant predictor of purchasing behavior. Using behavior is mainly determined by income, perceived consumer effectiveness and age, while recycling behavior is strongly influenced by using behavior. These findings have policy implications and improve understanding of green consumer behavior in China.