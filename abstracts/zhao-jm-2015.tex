This article investigates the effectiveness of China’s “Rural Appliance Rebate” (RAR) program in boosting rural appliance consumption during the latest economic recession. Based on two rounds of rural household survey data in China, we adopt a difference-in-difference matching approach to assess the extent to which these rebates induced greater rural appliance consumption than they otherwise would have. Our analysis finds that the RAR program boosted subsidized home appliance consumption, but its stimulus effects are driven mainly by cell phones, no significant impacts are detected on refrigerators and color TVs, and its impacts on aggregate home appliance consumption are not pronounced on the whole. Our investigation into the heterogeneity of these effects suggests that, for low income households, the RAR program prominently boost their aggregate home appliance consumption.