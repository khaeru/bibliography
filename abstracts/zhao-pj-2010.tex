 A key issue in the development of China's growing megacities in the transport-related environmental costs due to rapid urban expansion. In light of this issue, the authors examine the impact of urban form on commuting patterns on the city fringe of Beijing. Based on household-survey data, the results of the analysis suggest that the forms of land use adopted in the suburbs have a significant impact on commuting distance when a worker's socioeconomic characteristics and the level of transport accessibility are taken into account. Sprawling expansion, characterized by a low degree of self-contained development and low-density land use, tends to increase the need for long-distance commuting to the central urban area. In contrast, compact urban development in the suburbs, particularly in the peripheral constellations of Beijing, would reduce the probability of long-distance commuting. The current trend in improving transport accessibility on the city fringe is likely to lead to further long-distance commuting. In particular, huge road projects could cause more traffic congestion in the centre. The findings suggest that land-development management on the city fringe could have significant implications with respect to containing the dramatic costs to the environment entailed by transportation in the context of the rapid process of motorization. Reducing travel needs through the integration of land use and transport-infrastructure provision is likely to be the key to sustainable urban expansion. 