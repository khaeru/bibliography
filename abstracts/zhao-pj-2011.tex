Our existing knowledge of the links between urban growth and commuting patterns are dominated by cases from developed countries. This paper examines the impact of urban growth on workers' commutes using the case of Beijing, which is undergoing rapid economic and spatial restructuring. The results of an analysis of household survey data show that clustered and compact urban development in planned sub‐centres is likely to reduce suburban workers' need for a long‐distance commute to the city centre when the workers' socio‐economic characteristics, the level of transport accessibility and household preferences for residential location are taken into account. Workers employed in the manufacturing sector tend to have shorter commutes and travel within the planned suburban sub‐centres. This reveals that the decentralization of employment in the manufacturing sector provides more opportunities to enhance the spatial matches between household residential and job location choices. Household preferences for residential location have an effect on commuting patterns, and high‐income workers are likely to accept longer commutes in order to fulfil their residential preferences. Dramatic urban restructuring, in conjunction with changes in lifestyle, is creating new commuting patterns in the rapidly growing cities of China. Resumen El estado actual del conocimiento sobre los vínculos entre el crecimiento urbanístico y los patrones de desplazamiento al trabajo está dominado por ejemplos de países desarrollados. Este artículo examina el impacto del crecimiento urbano en los desplazamientos al trabajo en una ciudad como Pekín, la cual está siendo objeto de una rápida reestructuración económica y espacial. Los resultados del análisis de datos de un muestreo de hogares mostró que es probable que un desarrollo urbano compacto y agrupado en torno a sub‐centros planificados a propósito de este modo reduzca la necesidad de largos desplazamientos al trabajo hasta el centro de la ciudad, si se tienen en cuenta las características socioeconómicas de los trabajadores, la facilidad de acceso a transporte y las preferencias de cada hogar en cuanto a la zona residencial. Los trabajadores empleados en el sector manufacturero tienden a realizar desplazamientos más cortos y a viajar dentro de los sub‐centros suburbanos planificados. Esto revela que la descentralización del empleo en el sector manufacturero proporciona más oportunidades de mejorar la correspondencia espacial entre la elección de la zona residencial y la del puesto de trabajo. Las preferencias familiares en cuanto a la zona residencial tienen un efecto en los patrones de desplazamiento al trabajo, siendo más probable que los empleados con ingresos más altos estén dispuestos a realizar desplazamientos más largos a cambio de satisfacer sus preferencias residenciales. Una dramática reestructuración urbana, junto con cambios en el estilo de vida, está creando nuevos patrones de desplazamiento al trabajo en las ciudades chinas de rápido crecimiento.