Electric vehicles (EVs) have high energy efficiency and low pollutant and greenhouse gas (GHG) emissions compared with conventional internal combustion engine vehicles (ICEVs). This study examines the economic competitiveness of battery electric vehicles (BEVs) in the Chinese market. BEVs are compared with ICEVs using benefit-cost analyses from the perspectives of consumers, society and GHG emissions. A life-cycle cost model is developed to evaluate the lifetime cost of a vehicle. The results show that, with central government subsidies, the BEV life-cycle private cost (LCPC) is about 1.4 times higher than comparable ICEVs. Central government subsidies on BEVs will not be cost effective and efficient unless the annual external cost reduction from using BEV reaches $2500 for a compact vehicle or $3600 for a multi-purpose vehicle. That total cost level would imply a carbon cost of more than $2100 per ton. The current life-cycle external cost reductions from using BEV are around $2000–$2300, which are smaller than government subsidies or LCPC differences between BEV and ICEV. Further projections show that BEVs likely will not be economically competitive in the Chinese market before 2031.