Vehicles have recently overtaken coal to become the largest source of air pollution in urban China. Research on mobile sources of pollution has foundered due both to inaccessibility of Chinese data on health outcomes and strong identifying assumptions. To address these, we collect daily ambulance call data from the Beijing Emergency Medical Center and combine them with an idiosyncratic feature of a driving restriction policy in Beijing that references the last digit of vehicles’ license plate numbers. Because the number 4 is considered unlucky by many in China, it tends to be avoided on license plates. As a result, days on which the policy restricts license plates ending in 4 unintentionally allow more vehicles in Beijing. Leveraging this variation, we find that traffic congestion is indeed 22\% higher on days banning 4 and that 24-hour average concentration of NO2 is 12\% higher. Correspondingly, these short-term increases in pollution increase ambulance calls by 12\% and 3\% for fever and heart-related symptoms, while no effects are found for injuries. These findings suggest that traffic congestion has substantial health externalities in China but that they are also responsive to policy. 
