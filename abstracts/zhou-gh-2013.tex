China has promoted the use of electric vehicles vigorously since 2009; the program is still in its pilot phase. This study investigates the development of electric vehicle use in China from the perspectives of energy consumption and greenhouse-gas (GHG) emissions. Energy consumption and \{GHG\} emissions of plug-in hybrid electric vehicles (PHEVs) and pure battery electric vehicles (BEVs) are examined on the level of the regional power grid in 2009 through comparison with the energy consumption and \{GHG\} emissions of conventional gasoline internal combustion engine vehicles. The life-cycle analysis module in Tsinghua-LCAM, which is based on the \{GREET\} platform, is adopted and adapted to the life-cycle analysis of automotive energy pathways in China. Moreover, medium term (2015) and long term (2020) energy consumption and greenhouse-gas emissions of \{PHEVs\} and \{BEVs\} are projected, in accordance with the expected development target in the Energy Efficient and Alternative Energy Vehicles Industry Development Plan (2012–2020) for China. Finally, policy recommendations are provided for the proper development of electric vehicle use in China.