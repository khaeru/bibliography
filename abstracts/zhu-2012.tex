China has recently become the largest auto market in the world due to an explosive growth in private car ownership. On one hand, Chinese policymakers must develop sustainable transportation policies including those of car demand reduction; on the other, they must continue to promote the auto industry in the name of economic growth. Understanding the cultural currents behind car ownership, future car growth, and driver behavior is critical for developing responsive and effective policies. However, little is known regarding the attitudes and socio-cultural context surrounding the explosive growth in car ownership. A survey of Chinese college students – consumers with greater future purchasing power – was conducted to understand student attitudes, social norms, and aspiration for car ownership. A strong desire for car ownership among the participants supports the assertion that rapid car growth is likely to continue, but most likely in smaller cities and rural areas. Moreover, perceived psychosocial values of car ownership such as feelings of freedom and control are more likely to be framed by the students’ immediate social environment. The psychosocial valuations dominate the aspiration for car ownership at a level greater than the instrumental valuations of car ownership—for example, speed or convenience of using a private car. Understanding and changing consumers’ psychosocial valuation of cars is increasingly critical in curbing future growth in car ownership and use.
