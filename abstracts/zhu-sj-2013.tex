Traffic congestion has caused huge economic loss and environmental pollution every year. As a demand management policy to reduce congestion, vehicle ownership quota system that directly controls the number of vehicles on the road has recently been adopted in some metropolitan areas including Beijing and Shanghai. When it comes to implementation of quota system, Beijing uses the plate lottery system, so that everyone interested in owning a vehicle can participate and there's no monetary transaction in the process. Shanghai, on the other hand, uses the plate auction system and participants bid for the limited number of vehicle plates available. This paper aims at building a theoretical model that quantitatively analyzes the benefits of such policies. This study extends the joint decision model of vehicle ownership and mileage model, and applied compensating variation method to measure the net social impact change of the different quota systems. Under this proposed framework, a numerical demonstration is conducted.
